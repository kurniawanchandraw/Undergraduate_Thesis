\pagenumbering{arabic}\setcounter{page}{1}
\chapter{PENDAHULUAN}

\section{Latar Belakang}
Dalam ranah statistika spasial dan ekonometrika, tantangan fundamental yang terus dihadapi adalah pemodelan ketidakstasioneran spasial, yaitu suatu kondisi ketika parameter hubungan antarvariabel tidak bersifat konstan melainkan bervariasi melintasi ruang geografis. Pendekatan konvensional yang dominan selama dua dekade terakhir, yaitu \emph{Geographically Weighted Regression} (GWR), beroperasi di atas asumsi dasar bahwa heterogenitas spasial dapat ditangkap melalui fungsi pembobot (\emph{kernel}) yang bersifat tetap dan deterministik. Asumsi ini, yang berakar pada Hukum Pertama Geografi Tobler, menyatakan bahwa
``segala sesuatu saling berhubungan, tetapi hal-hal yang dekat lebih berhubungan daripada hal-hal yang jauh" \citep{fotheringham2002}. Namun, dalam kenyataannya, data modern semakin kompleks sehingga asumsi kaku mengenai peluruhan jarak (\emph{distance decay}) yang monotonik dan isotropik ini menjadi batasan metodologis yang serius. 

Penelitian ini tidak hanya berangkat dari fenomena empiris, melainkan dari kebutuhan mendesak untuk merevisi arsitektur matematis dari mekanisme pembobotan itu sendiri. GWR klasik, meskipun elegan dalam kesederhanaannya, mengalami apa yang disebut sebagai bias induktif (\emph{inductive bias}) yang sempit. Metode ini memaksakan struktur lingkungan Euclidean pada data yang seringkali berada pada Manifold non-Euclidean, seperti jaringan transportasi perkotaan, kepulauan yang terpisah laut, atau interaksi sosial-ekonomi yang tidak terikat jarak fisik semata. Ketergantungan pada kernel tetap (seperti \emph{Gaussian} atau \emph{exponential}) dengan bandwidth yang seringkali global atau adaptif secara terbatas, gagal menangkap dinamika dependensi spasial yang bersifat asimetris dan multi-skala. \citep{murakami2021}

Oleh karena itu, penelitian ini mengajukan konsepsi metodologis bahwa validitas inferensi dalam regresi spasial tidak lagi cukup hanya dengan mengkalibrasi \emph{bandwidth} pada kernel tetap. Diperlukan transisi menuju \emph{adaptive spatial weighting} yang dipelajari secara \emph{end-to-end} dari data itu sendiri. Di sinilah \emph{Graph Neural Networks} (GNN), dan secara spesifik \emph{Graph Attention Networks} (GAT), menawarkan solusi teoretis yang \emph{robust}. Dengan merepresentasikan unit spasial sebagai simpul (\emph{nodes}) dalam sebuah graf dan interaksi spasial sebagai sisi (\emph{edges}), GAT memungkinkan model untuk mempelajari bobot interaksi ($w_{ij}$) berdasarkan kesamaan fitur dan konteks topologi, bukan sekadar jarak geometris. \citep{velickovic2018} Sistematika pembobotan ini bersifat dinamis, kontekstual, dan mampu menangkap pola non-monotonik yang lebih realistis dalam hubungan spasial. Model GNN menghasilkan keluaran berupa skor perhatian (\emph{attention scores}) yang akan menjadi bobot spasial adaptif dalam kerangka GWR melalui fungsi aktivasi \emph{softmax}. Oleh karena itu, arsitektur ini diberi nama \emph{Graph Attention-based Geographically Weighted Regression} (GA-GWR) karena kemiripannya dengan GAT dalam hal mekanisme pembobotan adaptif.

Perkembangan terkini di bidang \emph{geospatial artificial intelligence} (GeoAI) menghadirkan pendekatan \emph{machine learning} yang dapat mempelajari pola nonlinear secara adaptif. Salah satunya adalah \emph{Geographically Weighted Artificial Neural Network} (GWANN) yang dikembangkan oleh \citet{hagenauer2022gwann} serta \emph{Spatial and Attribute Neural Network Weighted Regression} (SANNWR) oleh \citet{ni2022sannwr}. Keduanya berupaya memanfaatkan jaringan saraf tiruan untuk mengestimasi fungsi pembobot spasial yang kompleks. Lebih lanjut, \citet{yin2024gnnwr} memperkenalkan model \emph{Geographically Neural Network Weighted Regression} (GNNWR) dan \emph{Geographically and Temporally Neural Network Weighted Regression} (GTNNWR), yang memadukan kerangka \emph{varying coefficient} dengan kemampuan pembelajaran representasi nonlinear dari \emph{neural networks}. Model tersebut menunjukkan peningkatan akurasi yang signifikan dalam memetakan hubungan spasial-temporal pada data lingkungan dan sosial.

Di sisi lain, \citet{zhu2022srgcnn} mengusulkan \emph{Spatial Regression Graph Convolutional Neural Networks} (SRGCNN) yang mengintegrasikan \emph{graph convolutional neural network} (GCN) ke dalam paradigma regresi spasial. Pendekatan ini memperlakukan data spasial sebagai graf, dengan simpul merepresentasikan lokasi dan sisi (\emph{edges}) merepresentasikan keterhubungan spasial antarunit observasi. Dengan mekanisme \emph{message passing} yang mengagregasi informasi dari tetangga terdekat, SRGCNN tidak hanya mampu menangani struktur data non-Euclidean, tetapi juga mendukung pembelajaran semi-terawasi (\emph{semi-supervised learning}) yang memungkinkan model belajar dari data yang sebagian tidak teramati. Pendekatan ini membuka peluang besar untuk memperkuat model regresi spasial konvensional melalui pembobotan berbasis graf.

Berdasarkan keterbatasan fungsi \emph{kernel} klasik dan juga perkembangan tersebut, penelitian ini berupaya mengintegrasikan prinsip pembobotan adaptif berbasis GNN ke dalam kerangka GWR. Namun, penggunaan \emph{kernel} terestimasi dengan GNN dapat menyebabkan suatu permasalahan yang dinamakan sebagai endogenitas. Endogenitas terjadi ketika
\begin{equation}
    \text{Cov}(w_{ij}, \varepsilon_i) \neq 0,
\end{equation}
dengan $w_{ij}$ adalah bobot spasial dan $\varepsilon_i$ adalah istilah kesalahan pada lokasi $i$. Kondisi ini melanggar asumsi klasik bahwa bobot spasial harus eksogen terhadap kesalahan, yang dapat mengakibatkan estimasi parameter yang bias dan tidak konsisten. Dengan demikian, model yang diusulkan tidak hanya mengakomodasi ketidakstasioneran spasial-temporal, tetapi juga memperbaiki validitas inferensi statistik dengan analisis asimtotik. Dalam konteks empiris, model ini akan diterapkan pada analisis spasial Umur Harapan Hidup (UHH) antarkabupaten/kota di Indonesia periode 2023--2025, guna mengidentifikasi faktor-faktor sosial-ekonomi yang memengaruhi disparitas UHH secara spasial-temporal. Dengan demikian, penelitian ini diharapkan dapat memberikan kontribusi signifikan baik dari segi metodologis maupun aplikatif dalam bidang statistika spasial dan ekonometrika.

\section{Pembatasan Masalah}
Penelitian ini berfokus pada pengembangan, validitas inferensial, dan penerapan model GA-GWR pada kasus Umur Harapan Hidup (UHH) antarkabupaten/kota di Indonesia periode 2023--2025. Adapun pembatasan masalah dalam penelitian ini adalah sebagai berikut.
\begin{enumerate}[label=(\alph*)]
    \item Model GA-GWR yang dikembangkan hanya mengakomodasi ketidakstasioneran spasial dan temporal melalui pembobotan adaptif berbasis GNN, tanpa mempertimbangkan aspek multikolinearitas spasial atau heteroskedastisitas spasial.
    \item Validitas inferensial model GA-GWR dianalisis melalui pendekatan asimtotik, dengan asumsi bahwa data yang digunakan memenuhi kondisi stasioneritas temporal dan independensi spasial.
    \item Studi kasus difokuskan pada analisis UHH antarkabupaten/kota di Indonesia periode 2023--2025, dengan menggunakan data sekunder dari BPS.
    \item Analisis dan komputasi dilakukan menggunakan bahasa pemrograman Python dengan pustaka \textit{PyTorch Geometric} untuk implementasi jaringan saraf graf.
\end{enumerate} 

\section{Tujuan dan Manfaat Penelitian}
Tujuan utama penelitian ini adalah mengembangkan model pembelajaran spasial-temporal berbasis graf untuk menganalisis variasi Umur Harapan Hidup (UHH) antarkabupaten/kota di Indonesia. Secara khusus, penelitian ini bertujuan untuk:
\begin{enumerate}
    \item mengonstruksi model \emph{Graph Attention-based Geographically Weighted Regression} (GA-GWR) sebagai perluasan dari model GWR yang mampu menangkap ketidakstasioneran spasial melalui pembobotan adaptif berbasis Graph Attention Networks;
    \item mengevaluasi validitas inferensial model melalui analisis asimtotik dan membandingkan kinerja model terhadap model konvensional (GWR dan regresi klasik) dalam hal ketepatan estimasi, stabilitas, dan interpretabilitas koefisien lokal; serta
    \item menerapkan model yang dikembangkan pada studi kasus UHH antarkabupaten/kota di Indonesia periode 2023--2025 untuk mengidentifikasi pola spasial dan faktor-faktor sosial-ekonomi yang berpengaruh.
\end{enumerate}
Secara praktis, hasil penelitian ini diharapkan dapat memberikan kontribusi bagi lembaga perencana kebijakan, khususnya Kementerian Kesehatan dan Badan Pusat Statistik, dalam memahami disparitas UHH regional dan merancang kebijakan pembangunan kesehatan yang adaptif terhadap kondisi spasial antarkabupaten/kota di Indonesia.

\section{Tinjauan Pustaka}
Penelitian mengenai regresi spasial dan spasial-temporal telah berkembang pesat dalam dua dekade terakhir. Model klasik seperti GWR \citep{fotheringham2002} dan GTWR \citep{huang2010} menjadi dasar utama dalam analisis hubungan spasial yang tidak stasioner. Namun, keterbatasannya pada asumsi linearitas dan bentuk kernel mendorong munculnya model berbasis kecerdasan buatan seperti GWANN \citep{hagenauer2022gwann}, SANNWR \citep{ni2022sannwr}, dan GNNWR \citep{yin2024gnnwr}. Di sisi lain, pendekatan \emph{graph-based learning} seperti SRGCNN \citep{zhu2022srgcnn} memperlihatkan potensi besar dalam memodelkan struktur dependensi spasial-temporal yang kompleks melalui pembelajaran representasi graf. Berangkat dari temuan tersebut, penelitian ini berupaya menggabungkan kekuatan metodologis antara regresi terboboti spasial-temporal dan pembelajaran berbasis graf guna mengembangkan model yang lebih adaptif, akurat, dan interpretatif.

\section{Metodologi Penelitian}
Metodologi yang digunakan dalam penelitian ini meliputi \emph{stress test} dan studi kasus. \emph{Stress test} adalah serangkaian eksperimen simulasi yang dirancang untuk menguji kinerja model GA-GWR di bawah berbagai kondisi data sintetis. Eksperimen ini mencakup variasi ukuran sampel, tingkat kebisingan, pola ketidakstasioneran spasial-temporal, dan struktur dependensi antarunit observasi. Tujuannya adalah untuk mengevaluasi ketepatan estimasi, stabilitas koefisien lokal, serta sensitivitas model terhadap parameter hiper seperti arsitektur GNN dan fungsi aktivasi. Hasil dari \emph{stress test} akan dibandingkan dengan model GWR klasik dan GTWR untuk menilai keunggulan relatif dari pendekatan yang diusulkan. Selain itu, analisis asimtotik akan dilakukan untuk memastikan validitas inferensial dari estimasi parameter dalam model GA-GWR.

Studi kasus akan diterapkan pada data Umur Harapan Hidup (UHH) antarkabupaten/kota di Indonesia periode 2023--2025. Data sekunder akan diperoleh dari Badan Pusat Statistik (BPS) dan Kementerian Kesehatan. Analisis akan melibatkan pemodelan hubungan antara UHH dengan variabel sosial-ekonomi seperti pendapatan per kapita, tingkat pendidikan, akses layanan kesehatan, dan faktor lingkungan. Model GA-GWR akan diimplementasikan menggunakan bahasa pemrograman Python dengan pustaka \textit{PyTorch Geometric} untuk membangun dan melatih jaringan saraf graf. Hasil analisis akan mencakup peta koefisien lokal, evaluasi kinerja model, serta interpretasi hasil dalam konteks kebijakan kesehatan regional.

\section{Sistematika Penulisan}
Pada penyusunan skripsi ini, penulis mengacu pada sistematika penulisan sebagai berikut.\\
\newline \textbf{BAB I PENDAHULUAN}
\par Bab ini berisi tentang latar belakang, pembatasan masalah, tujuan dan manfaat penelitian, tinjauan pustaka, metodologi penelitian, dan sistematika penulisan.\\
\newline \textbf{BAB II LANDASAN TEORI}
\par Bab ini membahas dasar-dasar teori yang digunakan, termasuk teori regresi spasial-temporal, jaringan saraf graf, dan pembelajaran semi-terawasi.\\
\newline \textbf{BAB III KERANGKA METODOLOGIS DAN ANALISIS ASIMTOTIK \emph{GRAPH ATTENTION-BASED GEOGRAPHICALLY WEIGHTED REGRESSION}}
\par Bab ini menjelaskan rancangan model GA-GWR, formulasi matematis, analisis asimtotik, dan analisis komputasi.\\
\newline \textbf{BAB IV STUDI KASUS}
\par Bab ini berisi studi simulasi untuk \emph{stress test} dan hasil implementasi model pada data kredit UMKM di Indonesia.\\
\newline \textbf{BAB V PENUTUP}
\par Bab ini menyajikan kesimpulan yang diperoleh dari hasil penelitian dan memberikan saran untuk penelitian selanjutnya.
