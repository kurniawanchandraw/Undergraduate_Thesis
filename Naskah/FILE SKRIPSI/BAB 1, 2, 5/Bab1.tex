\pagenumbering{arabic}\setcounter{page}{1}
\chapter{PENDAHULUAN}

\section{Latar Belakang}
Perekonomian Indonesia senantiasa menghadapi dinamika yang kompleks seiring dengan variasi spasial dan temporal yang terjadi di setiap wilayah. Fenomena inflasi, misalnya, tidak hanya dipengaruhi oleh faktor makroekonomi secara nasional, tetapi juga oleh karakteristik lokal seperti struktur industri, tingkat pendidikan, dan infrastruktur wilayah. Penelitian \citet{amri2020} menunjukkan bahwa model \emph{Space-Time Autoregressive} (STAR) dan \emph{Generalized STAR} (GSTAR) mampu menangkap keterkaitan spasial antarwilayah dalam peramalan indeks harga konsumen (IHK) di beberapa kota besar di Sumatra. Namun, kedua model tersebut masih bersifat linear dan global sehingga belum sepenuhnya mampu menggambarkan ketidakstasioneran hubungan ekonomi antarwilayah dan antarwaktu.

Dalam literatur ekonometrika spasial, ketidakstasioneran spasial (\emph{spatial heterogeneity}) dan ketergantungan spasial (\emph{spatial dependence}) menjadi dua aspek penting yang perlu diperhatikan \citep{anselin}. Untuk mengakomodasi hal tersebut, \citet{fotheringham2002} mengembangkan model \emph{Geographically Weighted Regression} (GWR) yang memungkinkan koefisien regresi bervariasi antar lokasi. Model ini kemudian diperluas oleh \citet{huang2010} menjadi \emph{Geographically and Temporally Weighted Regression} (GTWR) dengan menambahkan dimensi waktu, sehingga mampu menangkap dinamika spasial sekaligus temporal dalam satu kerangka analisis. Walaupun interpretatif, model GTWR masih bergantung pada asumsi bentuk fungsi kernel dan pemilihan \emph{bandwidth} yang bersifat subjektif serta tidak mampu merepresentasikan hubungan nonlinear yang kompleks antara jarak spasial dan bobot pembobotan.

Perkembangan terkini di bidang \emph{geospatial artificial intelligence} (GeoAI) menghadirkan pendekatan \emph{machine learning} yang dapat mempelajari pola nonlinear secara adaptif. Salah satunya adalah \emph{Geographically Weighted Artificial Neural Network} (GWANN) yang dikembangkan oleh \citet{hagenauer2022gwann} serta \emph{Spatial and Attribute Neural Network Weighted Regression} (SANNWR) oleh \citet{ni2022sannwr}. Keduanya berupaya memanfaatkan jaringan saraf tiruan untuk mengestimasi fungsi pembobot spasial yang kompleks. Lebih lanjut, \citet{yin2024gnnwr} memperkenalkan model \emph{Geographically Neural Network Weighted Regression} (GNNWR) dan \emph{Geographically and Temporally Neural Network Weighted Regression} (GTNNWR), yang memadukan kerangka \emph{varying coefficient} dengan kemampuan pembelajaran representasi nonlinear dari \emph{neural networks}. Model tersebut menunjukkan peningkatan akurasi yang signifikan dalam memetakan hubungan spasial-temporal pada data lingkungan dan sosial.

Di sisi lain, \citet{zhu2022srgcnn} mengusulkan \emph{Spatial Regression Graph Convolutional Neural Networks} (SRGCNN) yang mengintegrasikan \emph{graph convolutional neural network} (GCN) ke dalam paradigma regresi spasial. Pendekatan ini memperlakukan data spasial sebagai graf, dengan simpul merepresentasikan lokasi dan sisi (\emph{edges}) merepresentasikan keterhubungan spasial antarunit observasi. Dengan mekanisme \emph{message passing} yang mengagregasi informasi dari tetangga terdekat, SRGCNN tidak hanya mampu menangani struktur data non-Euclidean, tetapi juga mendukung pembelajaran semi-terawasi (\emph{semi-supervised learning}) yang memungkinkan model belajar dari data yang sebagian tidak teramati. Pendekatan ini membuka peluang besar untuk memperkuat model regresi spasial konvensional melalui pembobotan berbasis graf.

Berdasarkan perkembangan tersebut, penelitian ini berupaya mengintegrasikan prinsip pembobotan adaptif berbasis GNN sebagaimana pada GNNWR dan SRGCNN ke dalam kerangka regresi spasial-temporal seperti GTWR. Model yang diusulkan adalah \emph{Graph Neural Network–Geographically and Temporally Varying Coefficient} (GNN-GTVC) serta \emph{Graph Neural Network–Geographically and Temporally Weighted Regression} (GNN-GTWR) . Kedua model ini dirancang untuk menangkap dinamika spasial-temporal dengan lebih fleksibel melalui pembelajaran berbasis graf yang mengoptimasi matriks bobot. Dalam konteks empiris, model ini akan diterapkan pada analisis spasial-temporal inflasi antarprovinsi di Indonesia periode 2024–2025, guna memahami disparitas inflasi antarwilayah dan dinamika faktor-faktor ekonomi yang memengaruhinya. Dengan demikian, penelitian ini diharapkan dapat memberikan kontribusi metodologis dalam bidang ekonometrika spasial sekaligus implikasi kebijakan dalam pengendalian inflasi regional di Indonesia.

\section{Pembatasan Masalah}
Penelitian ini berfokus pada pengembangan dan penerapan model GNN-GTWR dan GNN-GTVC untuk menganalisis dinamika spasial-temporal inflasi antarprovinsi di Indonesia. Ruang lingkup penelitian dibatasi pada level provinsi sebagai unit analisis dengan periode pengamatan Januari 2024 hingga Agustus 2025. Variabel yang digunakan meliputi indikator makroekonomi utama seperti inflasi, upah minimum, tingkat pengangguran, dan indeks pembangunan manusia (IPM) yang bersumber dari Badan Pusat Statistik (BPS) dan Bank Indonesia (BI). Penelitian ini tidak mencakup analisis mikroekonomi individu maupun agregasi lintas negara, serta tidak membahas secara mendalam mekanisme kausalitas antarvariabel.

\section{Tujuan dan Manfaat Penelitian}
Tujuan utama penelitian ini adalah mengembangkan model pembelajaran spasial-temporal berbasis graf untuk menganalisis variasi inflasi antarwilayah di Indonesia. Secara khusus, penelitian ini bertujuan untuk:
\begin{enumerate}
    \item mengonstruksi model \emph{Graph Neural Network–Geographically and Temporally Weighted Regression} (GNN-GTWR) dan \emph{Graph Neural Network–Geogra-phically and Temporally Varying Coefficient} (GNN-GTVC) sebagai perluasan dari model GTWR dan VCM yang mampu menangkap ketidakstasioneran spasial dan temporal secara simultan melalui pembobotan adaptif berbasis GNN;
    \item menerapkan model yang dikembangkan pada studi kasus inflasi antarprovinsi di Indonesia periode 2024–2025 untuk mengidentifikasi pola spasial-temporal dan faktor-faktor ekonomi yang berpengaruh; dan
    \item mengevaluasi kinerja model terhadap model konvensional (GTWR dan GWR) dalam hal ketepatan estimasi, stabilitas, dan interpretabilitas koefisien lokal.
\end{enumerate}
Secara praktis, hasil penelitian ini diharapkan dapat memberikan kontribusi bagi lembaga perencana kebijakan, khususnya Bank Indonesia dan Kementerian Keuangan, dalam memahami dinamika inflasi regional dan merancang kebijakan pengendalian harga yang adaptif terhadap kondisi spasial-temporal.

\section{Tinjauan Pustaka}
Penelitian mengenai regresi spasial dan spasial-temporal telah berkembang pesat dalam dua dekade terakhir. Model klasik seperti GWR \citep{fotheringham2002} dan GTWR \citep{huang2010} menjadi dasar utama dalam analisis hubungan spasial yang tidak stasioner. Namun, keterbatasannya pada asumsi linearitas dan bentuk kernel mendorong munculnya model berbasis kecerdasan buatan seperti GWANN \citep{hagenauer2022gwann}, SANNWR \citep{ni2022sannwr}, dan GNNWR \citep{yin2024gnnwr}. Di sisi lain, pendekatan \emph{graph-based learning} seperti SRGCNN \citep{zhu2022srgcnn} dan GSTRGCN \citep{xiong2024gstrgct} memperlihatkan potensi besar dalam memodelkan struktur dependensi spasial-temporal yang kompleks melalui pembelajaran representasi graf. Berangkat dari temuan tersebut, penelitian ini berupaya menggabungkan kekuatan metodologis antara regresi terboboti spasial-temporal dan pembelajaran berbasis graf guna mengembangkan model yang lebih adaptif, akurat, dan interpretatif.

\section{Metodologi Penelitian}
Metodologi yang digunakan dalam penelitian ini meliputi studi literatur dan studi kasus. Studi literatur dilakukan untuk mengkaji teori dan penelitian terdahulu terkait regresi spasial-temporal, jaringan saraf graf, serta pendekatan pembelajaran semi-terawasi yang relevan dengan pengembangan model GNN-GTWR dan GNN-GTVC. Studi kasus dilakukan dengan menggunakan data sekunder dari BPS dan BI, yang mencakup 38 provinsi di Indonesia untuk periode Januari 2024 hingga Agustus 2025. Analisis dan komputasi dilakukan menggunakan bahasa pemrograman Python dengan pustaka \textit{PyTorch Geometric} untuk implementasi jaringan saraf graf, serta perangkat lunak statistik seperti R atau GeoDa untuk validasi spasial.

\section{Sistematika Penulisan}
Pada penyusunan skripsi ini, penulis mengacu pada sistematika penulisan sebagai berikut.\\
\newline \textbf{BAB I PENDAHULUAN}
\par Bab ini berisi tentang latar belakang, pembatasan masalah, tujuan dan manfaat penelitian, tinjauan pustaka, metodologi penelitian, dan sistematika penulisan.\\
\newline \textbf{BAB II LANDASAN TEORI}
\par Bab ini membahas dasar-dasar teori yang digunakan, termasuk teori regresi spasial-temporal, jaringan saraf graf, dan pembelajaran semi-terawasi.\\
\newline \textbf{BAB III PENGEMBANGAN PEMBOBOTAN BERBASIS JARINGAN SARAF GRAF UNTUK REGRESI SPASIAL-TEMPORAL}
\par Bab ini menjelaskan rancangan model GNN-GTVC dan GNN-GTWR, formulasi matematis, serta prosedur estimasi dan validasi model.\\
\newline \textbf{BAB IV STUDI KASUS}
\par Bab ini berisi hasil implementasi model pada data inflasi antarprovinsi di Indonesia, analisis spasial-temporal, serta perbandingan kinerja model.\\
\newline \textbf{BAB V PENUTUP}
\par Bab ini menyajikan kesimpulan yang diperoleh dari hasil penelitian dan memberikan saran untuk penelitian selanjutnya.
