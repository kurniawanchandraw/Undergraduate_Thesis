\chapter{METODE REGRESI TERBOBOTI GEOGRAFIS DENGAN PEMBOBOTAN BERBASIS JARINGAN SARAF GRAF}

\section{Kerangka Regresi Terboboti Geografis}

Regresi Terboboti Geografis (\emph{Geographically Weighted Regression}, GWR)
merupakan pengembangan dari model regresi linear klasik yang bertujuan untuk
mengakomodasi ketidakstasioneran spasial. Dalam banyak aplikasi spasial,
hubungan antara variabel respons dan kovariat tidak bersifat global, melainkan
bervariasi antar lokasi. Untuk menangkap fenomena tersebut, GWR mengizinkan
koefisien regresi berubah secara halus mengikuti lokasi geografis
\citep{fotheringham2002gwr}.

\subsection{Model Regresi Terboboti Geografis}

Secara matematis, GWR memodelkan hubungan lokal antara variabel respons dan
kovariat dengan memperkenalkan fungsi koefisien yang bergantung pada lokasi.

\begin{definisi}[Model Regresi Terboboti Geografis]
Misalkan $\{(y_i,\mathbf{x}_i,\mathbf{u}_i)\}_{i=1}^n$ adalah himpunan
pengamatan, dengan $y_i \in \mathbb{R}$ menyatakan variabel respons,
$\mathbf{x}_i \in \mathbb{R}^p$ vektor kovariat, dan
$\mathbf{u}_i \in \mathbb{R}^d$ koordinat spasial.
Model Regresi Terboboti Geografis didefinisikan sebagai
\[
y_i = \mathbf{x}_i^\top \boldsymbol{\beta}(\mathbf{u}_i) + \varepsilon_i,
\qquad i = 1,\dots,n,
\]
dengan $\boldsymbol{\beta}(\mathbf{u})$ merupakan fungsi koefisien regresi
yang berubah terhadap lokasi $\mathbf{u}$, dan $\varepsilon_i$ adalah galat
acak dengan $\mathbb{E}(\varepsilon_i)=0$ dan
$\mathrm{Var}(\varepsilon_i)=\sigma^2$.
\end{definisi}

Model ini dapat dipandang sebagai pendekatan lokal dari regresi linear,
di mana setiap lokasi memiliki parameter tersendiri yang diestimasi dengan
memberikan bobot lebih besar pada pengamatan yang berdekatan secara spasial.

\subsection{Estimasi \textit{Locally Weighted Least Squares}}

Estimasi parameter lokal pada GWR dilakukan menggunakan pendekatan
\emph{locally weighted least squares} (LWLS), yang merupakan bentuk khusus
dari regresi linear terboboti dengan bobot yang bergantung pada lokasi target.

\begin{definisi}[Estimator Locally Weighted Least Squares]
Untuk suatu lokasi $\mathbf{u}_0$, estimator LWLS dari
$\boldsymbol{\beta}(\mathbf{u}_0)$ didefinisikan sebagai solusi dari masalah
minimisasi
\[
\min_{\boldsymbol{\beta}}
\sum_{i=1}^n w_i(\mathbf{u}_0)
\bigl(y_i - \mathbf{x}_i^\top \boldsymbol{\beta}\bigr)^2.
\]
Dalam notasi matriks, estimator tersebut dapat dituliskan sebagai
\[
\widehat{\boldsymbol{\beta}}(\mathbf{u}_0)
=
\bigl(\mathbf{X}^\top \mathbf{W}(\mathbf{u}_0)\mathbf{X}\bigr)^{-1}
\mathbf{X}^\top \mathbf{W}(\mathbf{u}_0)\mathbf{y},
\]
dengan $\mathbf{W}(\mathbf{u}_0)$ merupakan matriks diagonal bobot spasial.
\end{definisi}

Estimator ini menunjukkan bahwa GWR secara struktural merupakan perluasan
dari regresi linear terboboti, dan secara teoritis berkaitan erat dengan regresi
nonparametrik berbasis kernel \citep{fan1996local}.

\subsection{Struktur Pembobotan Spasial}

Bobot spasial pada GWR umumnya ditentukan sebagai fungsi dari jarak antara
lokasi pengamatan dan lokasi target. Bobot tersebut dibangun melalui fungsi
kernel dan parameter bandwidth yang mengontrol tingkat kehalusan atau \textit{smoothing}
\citep{fotheringham2002gwr}.

\section{Inferensi Asimtotik pada Regresi Terboboti Geografis}

Inferensi statistik pada GWR didasarkan pada sifat asimtotik dari estimator LWLS.
Karena estimator bersifat lokal dan bergantung pada bandwidth, analisis
asimtotiknya mengikuti kerangka regresi nonparametrik.

\subsection{Asumsi Regularitas}

Untuk memperoleh hasil asimtotik, diperlukan beberapa asumsi regularitas
standar, antara lain:
\begin{enumerate}[label=(A\arabic*)]
    \item Fungsi koefisien $\boldsymbol{\beta}(\mathbf{u})$ kontinu dan memiliki
    turunan hingga pada lingkungan $\mathbf{u}_0$;
    \item Matriks $\mathbb{E}(\mathbf{x}_i\mathbf{x}_i^\top \mid \mathbf{u}_i=\mathbf{u}_0)$
    bersifat definit positif;
    \item Galat $\varepsilon_i$ bersifat independen dan identik terdistribusi
    dengan momen hingga orde dua;
    \item Bandwidth $h=h_n$ memenuhi $h_n \to 0$ dan $n h_n^d \to \infty$.
\end{enumerate}
Asumsi-asumsi ini merupakan analog dari asumsi standar dalam regresi
nonparametrik kernel \citep{fan1996local,vdvaart1998}.

\subsection{Konsistensi Estimator Lokal}

Di bawah asumsi regularitas di atas, estimator LWLS bersifat konsisten terhadap
parameter lokal $\boldsymbol{\beta}(\mathbf{u}_0)$, yaitu
\begin{equation}
    \widehat{\boldsymbol{\beta}}(\mathbf{u}_0)
    \xrightarrow{p}
    \boldsymbol{\beta}(\mathbf{u}_0),
    \qquad n \to \infty.
\end{equation}
Konsistensi ini merupakan konsekuensi dari sifat pelicinan lokal dan fakta bahwa
jumlah pengamatan efektif di sekitar $\mathbf{u}_0$ meningkat seiring
bertambahnya ukuran sampel \citep{fan1996local}.

\subsection{Teorema Limit Pusat Koefisien Lokal}

Lebih lanjut, estimator lokal memiliki distribusi limit normal.
Secara khusus, berlaku
\begin{equation}
\label{eq:clt_gwr}
    \sqrt{n h_n^d}
    \left(
        \widehat{\boldsymbol{\beta}}(\mathbf{u}_0)
        -
        \boldsymbol{\beta}(\mathbf{u}_0)
        -
        \mathrm{Bias}(\mathbf{u}_0)
    \right)
    \xrightarrow{d}
    \mathcal{N}\!\left(
        \mathbf{0},
        \boldsymbol{\Sigma}(\mathbf{u}_0)
    \right),
\end{equation}
dengan $\mathrm{Bias}(\mathbf{u}_0)$ merupakan bias orde kedua yang bergantung
pada bandwidth dan turunan fungsi koefisien
\citep{fan1996local,vdvaart1998}.

\subsection{Variansi Asimtotik dan Inferensi Lokal}

Matriks kovarians asimtotik $\boldsymbol{\Sigma}(\mathbf{u}_0)$ bergantung pada
struktur kernel, distribusi kovariat, dan variansi galat.
Pendekatan ini memungkinkan pembentukan interval kepercayaan asimtotik dan
pengujian hipotesis lokal terhadap komponen-komponen
$\boldsymbol{\beta}(\mathbf{u}_0)$, dengan menggunakan pendekatan normal
asimtotik \citep{vdvaart1998}.

Inferensi yang dihasilkan bersifat lokal dan valid secara asimtotik pada setiap
lokasi, dengan asumsi struktur pembobotan spasial bersifat tetap dan tidak
terestimasi dari data.


\section{Keterbatasan Pembobotan Spasial Konvensional}
\subsection{Bias Induktif pada Kernel Spasial}
\subsection{Implikasi terhadap Inferensi Statistik}

\section{Metode Pembobotan Berbasis Jaringan Saraf Graf pada Regresi Terboboti Geografis}
\subsection{Representasi Graf untuk Data Spasial}
\subsection{Formulasi Estimasi Bobot dengan Jaringan Saraf Graf}
\subsection{Properti Bobot Terestimasi}
\subsection{Penduga Regresi Terboboti Geografis dengan Bobot GNN}

\section{Implikasi Pembobotan Berbasis Jaringan Saraf Graf terhadap Inferensi}
\subsection{Permasalahan Inferensi dengan Bobot Terestimasi}
\subsection{Representasi Bahadur Penduga dengan Bobot Terestimasi}
\subsection{Kondisi Validitas Asimtotik}
\subsection{Peran Cross-Fitting dalam Menjaga Validitas Inferensi}

\section{Model Koefisien Bervariasi Spasial sebagai Generalisasi}
\subsection{\textit{Geographically Neural Network Weighted Regression} (GNNWR)}
\subsection{Model Koefisien Bervariasi Spasial}
\subsection{Pembobotan Berbasis Jaringan Saraf Graf pada Model Koefisien Bervariasi}
\subsection{Implikasi Inferensi pada Model Koefisien Bervariasi Spasial}
