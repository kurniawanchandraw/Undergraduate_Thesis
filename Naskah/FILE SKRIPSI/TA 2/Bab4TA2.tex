\chapter{STUDI KASUS}

Bab ini menyajikan validasi empiris dari kerangka metodologis GA-GWR yang telah dikembangkan pada Bab~\ref{chap:metodologi}. Validasi dilakukan melalui dua pendekatan komplementer, yaitu studi simulasi dalam kondisi terkontrol dan aplikasi pada data riil. Studi simulasi bertujuan untuk memverifikasi properti teoritis estimator GA-GWR, khususnya kemampuan metode dalam merekonstruksi koefisien spasial sejati dan akurasi prediksi di bawah berbagai kondisi data. Sementara itu, studi kasus pada data inflasi regional Indonesia mendemonstrasikan kinerja metode dalam konteks aplikasi praktis dengan kompleksitas data riil. Kedua studi ini dirancang untuk menjawab pertanyaan penelitian utama tentang peningkatan kinerja pembelajaran adaptif bobot spasial melalui jaringan saraf graf dibandingkan pendekatan konvensional berbasis \emph{kernel} tetap.

\section{Studi Simulasi}

Studi simulasi merupakan instrumen penting dalam validasi metode statistik, khususnya untuk metode yang melibatkan pembelajaran mesin seperti GA-GWR. Berbeda dengan aplikasi pada data riil ketika nilai parameter sejati tidak diketahui, simulasi memungkinkan kontrol penuh atas \emph{data generating process} (DGP) sehingga kinerja estimator dapat dievaluasi secara objektif melalui perbandingan dengan nilai parameter sejati. Dalam konteks penelitian ini, studi simulasi dirancang untuk menjawab pertanyaan tentang seberapa dekat estimator GA-GWR terhadap koefisien spasial sejati dan bagaimana performa prediksi metode ini pada data \emph{out-of-sample} dibandingkan dengan metode pembanding seperti GWR klasik. Untuk menjawab pertanyaan tersebut, dirancang serangkaian eksperimen simulasi dengan variasi kompleksitas struktur spasial dan karakteristik data. Bagian berikut menguraikan desain eksperimen, metrik evaluasi, serta interpretasi hasil simulasi.

\subsection{Desain Eksperimen Simulasi}

Eksperimen simulasi dirancang untuk menguji kinerja GA-GWR terhadap struktur spasial yang kompleks. Kompleksitas ini penting untuk mengevaluasi kemampuan pembelajaran adaptif \emph{kernel} spasial dalam menangkap pola non-stasioner yang tidak dapat direpresentasikan oleh \emph{kernel} konvensional. Desain eksperimen mencakup struktur data, model generasi data, skenario variasi, dan prosedur evaluasi.

Data simulasi dibangkitkan dalam format panel spasial seimbang (\emph{balanced spatial panel}), yang merepresentasikan observasi berulang pada lokasi-lokasi tetap sepanjang waktu. Struktur ini dipilih karena relevansinya dengan data ekonomi makro regional dan konsistensinya dengan kerangka teoritis pada Bab~\ref{chap:metodologi}. Secara formal, data terdiri dari $N_{\text{LOC}}$ lokasi spasial yang masing-masing diamati selama $T$ periode waktu, menghasilkan total $n = N_{\text{LOC}} \times T$ observasi. Lokasi spasial $\mathbf{u}_i = (u_i, v_i)^\top$ didistribusikan pada \emph{grid regular} $[0, L] \times [0, L]$ dengan $N_{\text{LOC}} = m \times m$ lokasi dan $m$ adalah dimensi grid. Penggunaan \emph{grid regular} memfasilitasi visualisasi hasil melalui \emph{heatmap} dan memastikan cakupan spasial yang seragam di seluruh domain. Untuk setiap lokasi $i$ dan periode $t$, dibangkitkan kovariat $x_{it,j} \sim \mathcal{N}(0, 1)$ untuk $j = 1, 2$, yang bersifat independen antar waktu dan lokasi. Struktur ini mengisolasi variasi spasial pada koefisien dari variasi pada kovariat.

Model \emph{data generating process} mengikuti spesifikasi GWR dengan koefisien yang bervariasi secara spasial, yaitu
\begin{equation}
    y_{it} = \beta_0(\mathbf{u}_i) + \beta_1(\mathbf{u}_i) x_{it,1} + \beta_2(\mathbf{u}_i) x_{it,2} + \varepsilon_{it}, \quad \varepsilon_{it} \sim \mathcal{N}(0, \sigma^2),
    \label{eq:sim_dgp}
\end{equation}
dengan $\beta_j(\mathbf{u}_i)$ adalah fungsi koefisien spasial sejati untuk kovariat ke-$j$. Aspek krusial dari desain ini adalah pemilihan fungsi $\beta_j(\cdot)$ yang mengandung pola spasial kompleks yang sulit diestimasi oleh \emph{kernel} konvensional tetapi dapat dipelajari oleh GNN. Secara eksplisit, fungsi koefisien yang digunakan adalah sebagai berikut. Fungsi konstanta atau \emph{intercept} $\beta_0(\mathbf{u})$ didefinisikan sebagai
\begin{equation}
    \beta_0(\mathbf{u}) = 2 + 2 \sin(6r),
\end{equation}
dengan $r = \mathrm{arctan2}(v - 5, u - 5)/\pi + 0.5$ adalah fungsi fase sudut yang dinormalisasi. Fungsi ini menggambarkan perubahan \emph{gradual} dengan variasi sudut yang kompleks di seluruh domain. Di sini, $r = \sqrt{(u - L/2)^2 + (v - L/2)^2}$ adalah jarak radial dari pusat grid ke lokasi $\mathbf{u}$.

Fungsi kemiringan untuk kovariat pertama $\beta_1(\mathbf{u})$ adalah
\begin{equation}
    \beta_1(\mathbf{u}) =
    \begin{cases}
        2{,}5, & \text{jika } u < v, \\
        -2{,}5, & \text{jika } u \geq v.
    \end{cases},
\end{equation}
yang menggambarkan perubahan diskontinu di sepanjang diagonal utama \emph{grid}, menciptakan tantangan bagi metode yang mengandalkan \emph{kernel} halus.

Fungsi kemiringan untuk kovariat kedua $\beta_2(\mathbf{u})$ diberikan oleh
\begin{equation}
    \beta_2(\mathbf{u}) = 2 + \sin(u/2) \times \mathbb{I}_{v > 5},
\end{equation}
dengan $\mathbb{I}_{\{\cdot\}}$ adalah fungsi indikator. Fungsi ini memperkenalkan variasi spasial yang bersyarat berdasarkan posisi vertikal $v$, menambah kompleksitas pola spasial yang harus dipelajari.

Untuk mengevaluasi \emph{robustness} dan skalabilitas metode, dilakukan variasi pada dua faktor kunci yang mempengaruhi kompleksitas masalah estimasi yaitu ukuran sampel dan tingkat \emph{noise}. Variasi ukuran sampel dilakukan dengan tiga konfigurasi grid. Konfigurasi pertama menggunakan $m = 12$ yang menghasilkan $N_{\text{LOC}} = 144$ lokasi dengan $T = 25$ periode sehingga $n = 3.600$ observasi sebagai skenario \emph{baseline}. Konfigurasi kedua menggunakan $m = 20$ yang menghasilkan $N_{\text{LOC}} = 400$ lokasi dengan $n = 10.000$ observasi untuk menguji skalabilitas dengan ukuran sampel menengah. Konfigurasi ketiga menggunakan $m = 30$ yang menghasilkan $N_{\text{LOC}} = 900$ lokasi dengan $n = 22.500$ observasi untuk mengevaluasi kinerja pada \emph{dataset} besar. Variasi ini menguji bagaimana kinerja relatif GA-GWR terhadap GWR klasik berubah seiring peningkatan ukuran sampel, yang relevan dengan teori asimtotik pada Bab~\ref{chap:metodologi}.

Untuk mengevaluasi sensitivitas metode GWR konvensional terhadap pemilihan \emph{kernel}, dilakukan perbandingan empat \emph{kernel} yang berbeda pada skenario simulasi \emph{baseline} yang sama. \emph{Kernel}-\emph{kernel} yang dievaluasi adalah Gaussian, Exponential, Tricube, dan Bisquare dengan \emph{bandwidth} ditentukan menggunakan aturan Silverman, yaitu
\begin{equation}
    h_{\text{Silverman}} = 0{,}9 \min(\hat{\sigma}, \text{IQR}/1{,}34) n^{-1/5},
\end{equation}
dengan $\hat{\sigma}$ adalah simpangan baku jarak antar lokasi dan IQR. Aturan Silverman dipilih karena konsistensinya dengan teori asimtotik estimasi \emph{bandwidth} optimal dalam konteks \emph{kernel density estimation} (KDE). Secara teoritis, aturan ini memberikan \emph{bandwidth} yang meminimalkan \emph{integrated squared error} (ISE) dalam asumsi distribusi populasi normal, menghasilkan \emph{rate konvergensi} optimal sebesar $\mathcal{O}(n^{-1/5})$ yang sesuai dengan teori GWR. Alternatif lain yaitu pendekatan \emph{Generalized Cross-Validation} (GCV) untuk pencarian \emph{bandwidth} optimal secara langsung memiliki kompleksitas komputasi yang tidak \emph{scalable}, yaitu $\mathcal{O}(n^4)$ untuk estimasi setiap \emph{bandwidth} kandidat, sehingga pada data panel besar seperti pada studi kasus dengan $n = 3.600$ akan menghasilkan beban komputasi yang besar. Oleh karena itu, aturan Silverman merupakan \emph{trade-off} yang masuk akal antara akurasi dan efisiensi komputasi. Hasil perbandingan \emph{kernel} untuk skenario $n = 3.600$ dan $\sigma = 0{,}5$ disajikan dalam Tabel~\ref{tab:sim_kernel_comparison}.

\begin{table}[htbp]
\centering
\caption{Perbandingan Kernel GWR Gaussian dan Exponential}
\label{tab:sim_kernel_comparison}
\begin{tabular}{lccccc}
\toprule
\textbf{\emph{Kernel}} & \textbf{$R^2$} & \textbf{RMSE} & \textbf{MAE} & \textbf{Rataan RMSE($\boldsymbol{\beta}$)} & \textbf{Waktu Pred. (s)} \\
\midrule
Gaussian & 0,9305 & 1,0301 & 0,7134 & 0,4130 & 27,68 \\
Exponential & 0,8634 & 1,4442 & 1,0853 & 0,6321 & 28,95 \\
\bottomrule
\end{tabular}
\end{table}

Hasil perbandingan \emph{kernel} menunjukkan perbedaan substansial dalam performa berbagai \emph{kernel} pada data simulasi. Model GWR dengan \emph{kernel} Gaussian menunjukkan kinerja terbaik dengan $R^2$ tertinggi sebesar 0,9305 dan RMSE terendah sebesar 1,0301. Sebaliknya, \emph{kernel} Exponential menunjukkan penurunan kinerja yang signifikan dengan $R^2$ hanya 0,8634 dan RMSE sebesar 1,4442. Hal ini mengindikasikan bahwa \emph{kernel} Gaussian lebih efektif dalam menangkap variasi spasial pada data simulasi ini dibandingkan \emph{kernel} Exponential.

Variasi tingkat \emph{noise} dilakukan dengan tiga tingkat standar deviasi galat. Tingkat pertama menggunakan $\sigma = 0{,}5$ yang merepresentasikan kondisi \emph{noise} rendah dengan \emph{signal-to-}\emph{noise} \emph{ratio} tinggi. Tingkat kedua menggunakan $\sigma = 1{,}0$ yang merepresentasikan kondisi \emph{noise} sedang dengan \emph{signal-to-}\emph{noise} \emph{ratio} moderat. Tingkat ketiga menggunakan $\sigma = 2{,}0$ yang merepresentasikan kondisi \emph{noise} tinggi dengan \emph{signal-to-}\emph{noise} \emph{ratio} rendah. Variasi ini mengevaluasi sensitivitas metode terhadap kontaminasi \emph{noise}, yang penting karena pembelajaran GNN dapat mengalami \emph{overfitting} pada \emph{noise} jika tidak diregularisasi dengan baik. Kombinasi faktorial dari kedua faktor menghasilkan $3 \times 3 = 9$ skenario eksperimen.

Untuk mensimulasikan kondisi prediksi riil dan menghindari kebocoran data (\emph{data leakage}), digunakan skema \emph{temporal train-test split}. Dari $T$ periode waktu total, 80\% periode awal dialokasikan untuk himpunan data latih, sementara 20\% periode akhir dialokasikan untuk himpunan data uji. Penting dicatat bahwa semua lokasi hadir di kedua himpunan, hanya periode waktunya yang berbeda. Skema ini mensimulasikan skenario peramalan (\emph{forecasting}) ketika model dilatih pada data historis dan digunakan untuk memprediksi periode mendatang pada lokasi yang sama, yang merupakan pengaturan realistis untuk aplikasi pada data panel spasial dalam praktik.

Model GA-GWR dievaluasi dengan tiga arsitektur jaringan saraf graf berbeda sebagai \emph{backbone} untuk pembelajaran bobot spasial adaptif, yaitu \emph{Graph Attention Network} (GAT), \emph{Graph Convolutional Network} (GCN), dan \emph{GraphSAGE}. Untuk ketiga arsitektur, digunakan konfigurasi yang konsisten dalam implementasi untuk memastikan perbandingan yang adil. Konfigurasi \emph{hyperparameter} utama untuk model GA-GWR disajikan dalam Tabel~\ref{tab:ga_gwr_config}.

\begin{table}[htbp]
\centering
\caption{Konfigurasi Hyperparameter Model GA-GWR}
\label{tab:ga_gwr_config}
\begin{tabular}{ll}
\toprule
\textbf{Komponen} & \textbf{Spesifikasi} \\
\midrule
Jumlah lapisan jaringan & 3 lapisan \\
Dimensi lapisan tersembunyi & 64, 32, dan 16 unit \\
Fungsi aktivasi & ReLU (\emph{Rectified Linear Unit}) \\
Tingkat Dropout & 0,25 \\
Algoritma optimasi & Adam \\
Laju pembelajaran & 0,001 \\
Maksimal epoch & 500 \\
Mekanisme \emph{early stopping} & Aktif dengan \emph{patience} sebesar 30 epoch \\
Skema \emph{cross-fitting} & 3-\emph{fold} \emph{temporal cross-fitting} \\
\bottomrule
\end{tabular}
\end{table}

Fungsi kerugian yang dioptimalkan selama pelatihan model merupakan kombinasi multikomponen sebagai berikut. Fungsi kerugian utama adalah \emph{Mean Squared Error} (MSE) antara prediksi respons dan nilai target, yang dihitung dengan rumus:
\begin{equation}
    \text{MSE} = \frac{1}{n_{\text{train}}} \sum_{i=1}^{n_{\text{train}}} (y_i - \hat{y}_i)^2.
\end{equation}

Regularisasi entropi diterapkan pada bobot spasial yang dipelajari untuk mendorong distribusi bobot yang lebih merata. Regularisasi ini dihitung sebagai:
\begin{equation}
    L_{\text{entropy}} = -\sum_{j} w_j \log(w_j),
\end{equation}
dengan bobot $w_j$ adalah bobot spasial untuk tetangga ke-$j$.

Regularisasi kelancaran Laplacian diterapkan untuk memastikan bobot spasial yang berdekatan tidak berbeda terlalu tajam. Regularisasi ini dihitung sebagai:
\begin{equation}
    L_{\text{Laplacian}} = \sum_{(i,j) \in E} (w_i - w_j)^2,
\end{equation}
dengan $E$ adalah himpunan tepi pada graf tetangga.

Regularisasi monotonisitas \emph{kernel} diterapkan untuk menjaga penurunan bobot dengan jarak. Regularisasi ini dihitung sebagai:
\begin{equation}
    L_{\text{monotonicity}} = \sum_{i,j} \max(0, w_i - w_j),
\end{equation}
untuk tetangga $i$ dan $j$ dengan jarak $d_i < d_j$.

Regularisasi \emph{smoothness} koefisien diterapkan untuk memanfaatkan kontinuitas spasial koefisien. Regularisasi ini dihitung sebagai:
\begin{equation}
    L_{\text{coef\_smoothness}} = \sum_{(i,j) \in E} (\hat{\beta}(\mathbf{u}_i) - \hat{\beta}(\mathbf{u}_j))^2.
\end{equation}

\subsection{Evaluasi Akurasi Estimasi Titik}

Evaluasi akurasi estimasi bertujuan menjawab pertanyaan fundamental yaitu seberapa dekat estimator $\hat{\boldsymbol{\beta}}(\mathbf{u})$ terhadap koefisien spasial sejati $\boldsymbol{\beta}^*(\mathbf{u})$. Dalam konteks GWR, terdapat dua aspek akurasi yang perlu dievaluasi yaitu akurasi estimasi parameter lokal dan akurasi prediksi respons. Evaluasi dilakukan melalui serangkaian metrik kuantitatif yang mengukur kesalahan estimasi baik pada level parameter maupun prediksi.

Akurasi estimasi koefisien lokal diukur melalui \emph{Root Mean Squared Error} (RMSE) yang dihitung untuk setiap komponen koefisien $\beta_j(\mathbf{u})$ dengan $j = 0, 1, 2$ yang merepresentasikan konstanta dan dua koefisien regresi. Untuk $N_{\text{LOC}}$ lokasi unik, RMSE untuk koefisien $j$ didefinisikan sebagai
\begin{equation}
    \text{RMSE}(\hat{\beta}_j) = \sqrt{\frac{1}{N_{\text{LOC}}} \sum_{i=1}^{N_{\text{LOC}}} \left(\hat{\beta}_j(\mathbf{u}_i) - \beta_j^*(\mathbf{u}_i)\right)^2}.
\end{equation}
Metrik ini mengukur kesalahan estimasi rata-rata di seluruh domain spasial. Nilai RMSE yang lebih rendah mengindikasikan estimator yang lebih akurat dalam merekonstruksi pola spasial sejati. Sebagai agregasi, dihitung pula Mean RMSE sebagai rata-rata aritmatik dari RMSE ketiga koefisien yang memberikan ukuran kesalahan keseluruhan, yaitu
\begin{equation}
    \overline{\text{RMSE}} = \frac{1}{3} \sum_{j=0}^{2} \text{RMSE}(\hat{\beta}_j).
\end{equation}
Selain metrik global, untuk memahami pola spasial kesalahan estimasi, dihitung juga bias lokal untuk setiap koefisien yang didefinisikan sebagai
\begin{equation}
    \text{Bias}_j(\mathbf{u}_i) = \hat{\beta}_j(\mathbf{u}_i) - \beta_j^*(\mathbf{u}_i).
\end{equation}
Peta bias menunjukkan tempat metode cenderung melakukan \emph{overestimation} atau \emph{underestimation}, memberikan wawasan tentang karakteristik \emph{spatial smoothing} yang dihasilkan oleh masing-masing \emph{kernel} dan arsitektur pembelajaran.

Akurasi prediksi respons pada data uji dievaluasi menggunakan tiga metrik standar yang saling melengkapi. Pertama, koefisien determinasi $R^2$ mengukur proporsi variansi yang dijelaskan oleh model:
\begin{equation}
    R^2 = 1 - \frac{\sum_{i=1}^{n_{\text{test}}} (y_i - \hat{y}_i)^2}{\sum_{i=1}^{n_{\text{test}}} (y_i - \bar{y})^2},
\end{equation}
dengan nilai lebih tinggi mendekati satu mengindikasikan kecocokan yang lebih baik. Kedua, \emph{Root Mean Squared Error} (RMSE) prediksi mengukur efek kesalahan prediksi rata-rata dalam satuan asli variabel respons:
\begin{equation}
    \text{RMSE}_y = \sqrt{\frac{1}{n_{\text{test}}} \sum_{i=1}^{n_{\text{test}}} (y_i - \hat{y}_i)^2}.
\end{equation}
Ketiga, \emph{Mean Absolute Error} (MAE) memberikan ukuran kesalahan yang lebih \emph{robust} terhadap pencilan:
\begin{equation}
    \text{MAE} = \frac{1}{n_{\text{test}}} \sum_{i=1}^{n_{\text{test}}} |y_i - \hat{y}_i|.
\end{equation}

Hasil eksperimen pada skenario baseline dengan $n = 3.600$ observasi dan $\sigma = 0{,}5$ disajikan dalam Tabel~\ref{tab:sim_baseline} di bawah ini.

\begin{table}[htbp]
\centering
\caption{Perbandingan Performa Prediksi Model ($n = 3.600$, $\sigma = 0.5$)}
\label{tab:sim_baseline}
\begin{tabular}{lccc}
\toprule
\textbf{Model} & \textbf{$R^2$} & \textbf{RMSE} & \textbf{MAE} \\
\midrule
OLS & 0,3358 & 3,1846 & 2,5608 \\
GWR - Gaussian & 0,9305 & 1,0301 & 0,7134 \\
GA-GWR (GAT) & \textbf{0,9806} & \textbf{0,5441} & \textbf{0,4285} \\
\bottomrule
\end{tabular}
\end{table}

\noindent Lebih lanjut, tabel ini membandingkan kinerja OLS, GWR klasik dengan \emph{kernel} Gaussian dan \emph{bandwidth} yang dihitung melalui, dan GA-GWR dengan \emph{backbone} GraphSAGE yang menunjukkan efisiensi komputasi terbaik. 

\begin{table}[htbp]
\centering
\caption{RMSE Estimasi Koefisien Spasial}
\label{tab:sim_beta_rmse}
\begin{tabular}{lcccc}
\toprule
\textbf{Model} & \textbf{RMSE($\beta_0$)} & \textbf{RMSE($\beta_1$)} & \textbf{RMSE($\beta_2$)} & \textbf{Mean RMSE} \\
\midrule
OLS & 1,5033 & 2,4936 & 0,5081 & 1,5017 \\
GWR - Gaussian & 0,5901 & 0,5291 & 0,1197 & 0,4130 \\
GA-GWR (GAT) & \textbf{0,117} & \textbf{0,1149} & \textbf{0,117} & \textbf{0,116} \\
\bottomrule
\end{tabular}
\end{table}

\noindent Berdasarkan tabel di atas, GA-GWR menunjukkan keunggulan signifikan dalam akurasi prediksi dibandingkan GWR klasik dan OLS. Model GA-GWR mencapai $R^2$ sebesar 0,9806, jauh lebih tinggi dibandingkan GWR klasik (0,9305) dan OLS (0,3358). Demikian pula, RMSE dan MAE GA-GWR (0,5441 dan 0,4285) secara substansial lebih rendah daripada GWR klasik (1,0301 dan 0,7134) dan OLS (3,1846 dan 2,5608). Hasil ini menegaskan kemampuan GA-GWR dalam menangkap variasi spasial yang kompleks dengan akurasi prediksi yang baik. Visualisasi spasial koefisien sejati dibandingkan dengan estimasi dari metode terbaik ditampilkan dalam Gambar~\ref{fig:sim_spatial_coef}. 

\begin{figure}[htbp]
\centering
\includegraphics[width=\textwidth]{GAMBAR/Simulasi_02_Heatmap_Coefficients.pdf}
\caption{Perbandingan koefisien spasial sejati (baris atas), estimasi GWR klasik (baris tengah), dan estimasi GA-GWR (baris bawah) untuk $\beta_0$ (kolom kiri), $\beta_1$ (kolom tengah), dan $\beta_2$ (kolom kanan). GA-GWR menangkap pola spasial dengan lebih akurat termasuk diskontinuitas tajam pada $\beta_1$.}
\label{fig:sim_spatial_coef}
\end{figure}

Perbandingan arsitektur GNN sebagai \emph{backbone} GA-GWR disajikan dalam Tabel~\ref{tab:sim_backbone}. Hasil menunjukkan bahwa ketiga arsitektur memberikan performa prediksi dan estimasi koefisien yang sangat serupa, dengan GraphSAGE menunjukkan keunggulan dalam efisiensi komputasi.

\begin{table}[htbp]
\centering
\caption{Perbandingan Arsitektur GNN sebagai \emph{Backbone} GA-GWR}
\label{tab:sim_backbone}
\begin{tabular}{lcccc}
\toprule
\textbf{\emph{Backbone}} & \textbf{$R^2$} & \textbf{RMSE} & \textbf{Rataan RMSE($\boldsymbol{\beta}$)} & \textbf{Waktu Pelatihan (s)} \\
\midrule
GAT & \textbf{0,9806} & \textbf{0,5438} & \textbf{0,1169} & 583,43 \\
GCN & 0,9806 & 0,5439 & 0,1172 & \textbf{432,29} \\
GraphSAGE & 0,9805 & 0,5450 & 0,1206 & 527,58 \\
\bottomrule
\end{tabular}
\end{table}

Perbandingan arsitektur GNN menunjukkan bahwa ketiga arsitektur (GAT, GCN, dan GraphSAGE) menghasilkan performa prediksi dan estimasi koefisien yang hampir identik dengan nilai $R^2 = 0{,}9806$, RMSE $\approx 0{,}544$, dan Mearataan RMSE koefisien $\approx 0{,}1170$. Perbedaan signifikan terletak pada efisiensi komputasi, ketika GCN membutuhkan waktu pelatihan paling singkat (432,29 detik), disusul GraphSAGE (527,58 detik), dan GAT (583,43 detik). Keunggulan GAT dalam efisiensi komputasi yang kompetitif menjadi kriteria pemilihan utama untuk menggunakan arsitektur ini dalam studi kasus aplikasi pada data riil, dengan waktu pelatihan yang lebih cepat menguntungkan praktisnya implementasi tanpa mengorbankan akurasi.

Selanjutnya, analisis sensitivitas dilakukan untuk mengevaluasi dampak variasi ukuran sampel $N_{\text{LOC}}$ dan tingkat \emph{noise} $\sigma$ terhadap kinerja GA-GWR dengan \emph{backbone} GAT. Hasil analisis sensitivitas disajikan dalam Tabel~\ref{tab:sim_sensitivity} dan Gambar~\ref{fig:sim_sensitivity}.
\begin{table}[htbp]
\centering
\caption{Analisis Sensitivitas GA-GWR terhadap Ukuran Sampel dan Tingkat Noise}
\label{tab:sim_sensitivity}
\begin{tabular}{cccccc}
\toprule
\textbf{$N_{\text{LOC}}$} & \textbf{$\sigma$} & \textbf{$R^2$} & \textbf{RMSE} & \textbf{Rataan RMSE($\boldsymbol{\beta}$)} \\
\midrule
64 & 0,5 & 0,979 & 0,521 & 0,140 \\
64 & 1,0 & 0,920 & 1,041 & 0,280 \\
64 & 2,0 & 0,731 & 2,080 & 0,559 \\
\midrule
100 & 0,5 & 0,980 & 0,528 & 0,136 \\
100 & 1,0 & 0,923 & 1,057 & 0,281 \\
100 & 2,0 & 0,736 & 2,124 & 0,565 \\
\midrule
144 & 0,5 & 0,981 & 0,532 & 0,140 \\
144 & 1,0 & 0,926 & 1,058 & 0,279 \\
144 & 2,0 & 0,740 & 2,132 & 0,578 \\
\bottomrule
\end{tabular}
\end{table}

\noindent Berdasarkan tabel tersebut, terlihat bahwa peningkatan ukuran sampel $N_{\text{LOC}}$ dari 64 ke 144 memberikan peningkatan kecil pada $R^2$ dan peningkatan RMSE prediksi, tetapi dampaknya lebih signifikan pada rataan RMSE koefisien yang menunjukkan estimasi koefisien yang lebih akurat. Sebaliknya, peningkatan tingkat \emph{noise} $\sigma$ dari 0,5 ke 2,0 menyebabkan penurunan tajam pada $R^2$ dan peningkatan RMSE prediksi serta rataan RMSE koefisien, menandakan sensitivitas metode terhadap kontaminasi \emph{noise}. Visualisasi \emph{heatmap} dalam Gambar~\ref{fig:sim_sensitivity} memperjelas tren ini, dengan area berwarna lebih gelap menunjukkan kinerja yang lebih buruk pada kombinasi ukuran sampel kecil dan tingkat \emph{noise} tinggi.

\begin{figure}[htbp]
\centering
\includegraphics[width=\textwidth]{GAMBAR/Simulasi_07_Sensitivity_Analysis.pdf}
\caption{\emph{Heatmap} analisis sensitivitas GA-GWR.}
\label{fig:sim_sensitivity}
\end{figure}

Analisis sensitivitas menunjukkan bahwa GA-GWR dengan \emph{backbone} GAT mempertahankan kinerja prediksi yang sangat baik pada berbagai ukuran sampel, dengan peningkatan ukuran sampel memberikan manfaat tambahan pada akurasi estimasi koefisien. Namun, metode ini menunjukkan sensitivitas yang signifikan terhadap tingkat \emph{noise}, dengan penurunan kinerja yang substansial pada tingkat \emph{noise} tinggi. Temuan ini menyoroti pentingnya pengelolaan \emph{noise} dalam aplikasi praktis GA-GWR untuk memastikan hasil yang andal.

\subsection{Analisis Diagnostik Residual}

Analisis diagnostik residual merupakan komponen krusial dalam validasi asumsi model regresi dan identifikasi potensi kesalahan spesifikasi. Evaluasi difokuskan pada GA-GWR dengan \emph{backbone} GAT yang menunjukkan kinerja terbaik dalam studi simulasi \emph{baseline}. Analisis mencakup evaluasi terhadap distribusi probabilitas residual, struktur variansi, dan dependensi spasial residual.

Normalitas residual merupakan asumsi fundamental untuk validitas inferensi parametrik termasuk konstruksi interval kepercayaan dan uji hipotesis. Evaluasi dilakukan menggunakan tiga metode komplementer dengan hipotesis berupa
\begin{align*}
H_0 &: \text{residual berdistribusi normal}, \text{ lawan}\\
H_1 &: \text{residual tidak berdistribusi normal},
\end{align*}
dengan statistik uji berupa Shapiro-Wilk dengan statistik $W$ dan Kolmogorov-Smirnov dengan statistik $D$.

Analisis menghasilkan tabel hasil uji normalitas sebagaimana ditampilkan dalam Tabel~\ref{tab:sim_uji_normalitas}. Uji menunjukkan bahwa tidak terdapat cukup bukti untuk menolak hipotesis nol, mengindikasikan bahwa residual GA-GWR sangat konsisten dengan distribusi normal. Visualisasi Q-Q plot dalam Gambar~\ref{fig:sim_residual_diag} menunjukkan bahwa kuantil-kuantil empiris residual terletak sangat dekat dengan garis diagonal, dengan deviasi kecil hanya pada ekor ekstrem yang tidak substansial. Histogram residual menunjukkan distribusi yang hampir simetris dengan puncak di sekitar nol.

\begin{table}[htbp]
\centering
\caption{Hasil Uji Normalitas Residual GA-GWR}
\label{tab:sim_uji_normalitas}
\begin{tabular}{lccc}
\toprule
\textbf{Metode Uji} & \textbf{Statistik} & \textbf{Nilai-$p$} & \textbf{Keputusan} \\
\midrule
Shapiro-Wilk & $W = 0{,}9981$ & $0{,}6341$ & Gagal tolak $H_0$ \\
Kolmogorov-Smirnov & $D = 0{,}0195$ & $0{,}9427$ & Gagal tolak $H_0$ \\
\bottomrule
\end{tabular}
\end{table}

\noindent Residual GA-GWR secara keseluruhan sangat konsisten dengan asumsi normalitas. Implikasi praktis adalah bahwa prosedur inferensi parametrik dapat diandalkan untuk keperluan konstruksi interval kepercayaan dan uji hipotesis.

Heteroskedastisitas, yaitu variansi kesalahan yang tidak konstan di seluruh domain observasi, dapat menyebabkan estimator tidak efisien dan \emph{standard error} bias. Uji Breusch-Pagan digunakan untuk menguji apakah variansi residual bergantung secara sistematis pada prediktor dengan hipotesis berikut.
\begin{align*}
H_0 &: \sigma^2_{\varepsilon,i} = \sigma^2_{\varepsilon,j}, \forall i \ne j,\text{ lawan}\\
H_1 &:  \exists i \ne j \text{ sehingga } \sigma^2_{\varepsilon,i} \ne \sigma^2_{\varepsilon,j},
\end{align*}
dengan statistik uji berupa $\text{BP} = n \cdot R^2_{\text{auxiliary}}$ yang mengikuti distribusi $\chi^2$ dengan derajat kebebasan sebesar jumlah prediktor.

Analisis menghasilkan statistik Breusch-Pagan $\text{BP} = 1{,}9948$ dengan nilai-$p$ $= 0{,}5735$. Nilai-\(p\) yang jauh lebih besar dari $0{,}05$ mengindikasikan gagal menolak hipotesis nol. Nilai-$R^2$ dari regresi \emph{auxiliary} sebesar $0{,}0028$ menunjukkan bahwa proporsi variansi yang dijelaskan oleh prediktor terhadap kuadrat residual sangat kecil. \emph{Scatter plot} residual terhadap \emph{fitted values} dalam Gambar~\ref{fig:sim_residual_diag} tidak menunjukkan pola sistematis yang jelas.

Uji Moran's $I$ digunakan untuk mendeteksi keberadaan klaster atau dispersal spasial dalam residual sebagai perhitungan autokorelasi spasial dari residual. Matriks bobot spasial dikonstruksi berdasarkan $k$-\emph{nearest neighbors} dengan $k=5$, memastikan setiap lokasi memiliki tepat $5$ tetangga terdekat. Statistik Moran's $I$ dihitung sebagai
\begin{equation}
    I = \frac{n}{\sum_{i}\sum_{j}w_{ij}} \cdot \frac{\sum_{i}\sum_{j}w_{ij}e_i e_j}{\sum_{i}e_i^2}{,}
\end{equation}
dengan $e_i = y_i - \hat{y}_i$ adalah residual observasi ke-$i$ dan $w_{ij}$ adalah elemen matriks bobot spasial yang dinormalisasi baris. Uji dilakukan dengan hipotesis berikut.

\begin{align*}
H_0 &: I = 0 \quad \text{(tidak ada autokorelasi spasial)},\text{ lawan}\\
H_1 &: I \neq 0 \quad \text{(ada autokorelasi spasial)},
\end{align*}
dengan statistik uji berupa nilai $Z$ yang dihitung berdasarkan nilai $I$, ekspektasi di bawah $H_0$, dan varian.

Analisis menghasilkan Moran's $I = 0{,}2200$ dengan nilai $Z = 2{,}103$ dan nilai-$p$ $= 0{,}0354$. Nilai-$p$ kurang dari $0{,}05$ mengindikasikan adanya cukup bukti untuk menolak hipotesis nol pada tingkat signifikansi $\alpha = 0{,}05$. \emph{Heatmap} spasial residual dalam Gambar~\ref{fig:sim_residual_diag} menunjukkan bahwa residual positif dan negatif tidak terdistribusi secara acak sempurna, dengan beberapa pola klaster lokal. Terdapat autokorelasi spasial positif yang signifikan dalam residual. Namun, hal ini tidak menunjukkan kegagalan model secara keseluruhan, melainkan mengindikasikan adanya struktur spasial lokal yang belum ditangkap sepenuhnya oleh metode semi-parametrik. Interpretasi ini konsisten dengan model GWR dan GA-GWR yang dirancang untuk menangkap heterogenitas spasial pada skala besar, tidak pada setiap pola lokal mikro. Sesuai dengan batasan masalah yang dibahas pada Bab I, metode yang dikembangkan belum dirancang untuk menangkap autokorelasi spasial residual secara menyeluruh.

\begin{figure}[htbp]
\centering
\includegraphics[width=\textwidth]{GAMBAR/Simulasi_03_Residual_Diagnostics.pdf}
\caption{Analisis diagnostik residual GA-GWR (GraphSAGE backbone): (kiri atas) histogram residual standarisasi dengan overlay kurva normal, (kanan atas) Q-Q plot, (kiri bawah) residual vs fitted values, (tengah bawah) scale-location plot, (bawah kiri) residual autocorrelation, (bawah kanan) peta sebaran spasial residual.}
\label{fig:sim_residual_diag}
\end{figure}

\subsection{Validasi Inferensi Statistik}

Validasi inferensi statistik dilakukan melalui evaluasi \emph{coverage probability} interval prediksi. Interval prediksi dihitung berbasis \emph{residual training} dengan rumus $\text{PI}_{\alpha} = \hat{y} \pm z_{\alpha/2} \cdot \hat{\sigma}_{\text{residual}}$, dengan $z_{\alpha/2}$ adalah kuantil distribusi normal standar. Hal ini sah karena residual sudah dikonfirmasi berdistribusi normal berdasarkan hasil uji normalitas sebelumnya. Hasil evaluasi \emph{coverage probability} untuk berbagai tingkat kepercayaan disajikan dalam Tabel~\ref{tab:sim_coverage} dan divisualisasikan dalam Gambar~\ref{fig:sim_coverage}.

\begin{table}[htbp]
\centering
\caption{\emph{Coverage Probability} GA-GWR pada Berbagai Tingkat Kepercayaan}
\label{tab:sim_coverage}
\begin{tabular}{ccc}
\toprule
\textbf{Tingkat Kepercayaan} & \textbf{\emph{Coverage} Aktual} & \textbf{Deviasi} \\
\midrule
50\% & $44{,}2\%$ & $-5{,}8\%$ \\
80\% & $72{,}6\%$ & $-7{,}4\%$ \\
90\% & $82{,}5\%$ & $-7{,}5\%$ \\
95\% & $89{,}6\%$ & $-5{,}4\%$ \\
99\% & $97{,}5\%$ & $-2{,}5\%$ \\
\bottomrule
\end{tabular}
\end{table}

\begin{figure}[htbp]
\centering
\includegraphics[width=\textwidth]{GAMBAR/Simulasi_05_Coverage_Probability.pdf}
\caption{Evaluasi \emph{coverage probability} GA-GWR.}
\label{fig:sim_coverage}
\end{figure}

Berdasarkan tabel dan grafik tersebut, terlihat bahwa \emph{coverage probability} aktual GA-GWR sedikit di bawah tingkat kepercayaan yang diharapkan pada semua level. Deviasi negatif berkisar antara $-1{,}6\%$ hingga $-6{,}5\%$, menunjukkan bahwa interval prediksi cenderung sedikit terlalu sempit. Namun, deviasi ini relatif kecil, menandakan bahwa metode inferensi statistik berbasis residual cukup andal. Implikasi praktisnya adalah bahwa interval prediksi yang dihasilkan dapat digunakan dengan tingkat kepercayaan yang wajar, meskipun penyesuaian kecil mungkin diperlukan untuk aplikasi yang sangat kritis.

Selanjutnya, konvergensi pelatihan GA-GWR dievaluasi melalui kurva kerugian tiap \emph{epoch} untuk setiap \emph{fold} dalam 3-\emph{fold temporal cross-fitting}, sebagaimana ditampilkan dalam Gambar~\ref{fig:sim_loss}. 

\begin{figure}[htbp]
\centering
\includegraphics[width=\textwidth]{GAMBAR/Simulasi_06_Loss_vs_Epoch.pdf}
\caption{Kurva konvergensi pelatihan GA-GWR, kerugian tiap \emph{epoch} untuk setiap \emph{fold} dalam 3-\emph{fold temporal cross-fitting}.}
\label{fig:sim_loss}
\end{figure}

Berdasarkan grafik tersebut, terlihat bahwa kerugian memiliki tren penurunan setiap bertambahnya \emph{epoch}, menunjukkan proses pelatihan yang baik. Meskipun semikian, variansi kerugian antar-\emph{epoch} menunjukkan fluktuasi yang cukup besar, menandakan potensi ketidakstabilan dalam proses optimasi. Penggunaan mekanisme \emph{early stopping} membantu mencegah \emph{overfitting} dengan menghentikan pelatihan ketika kerugian pada data validasi tidak membaik selama 30 \emph{epoch} berturut-turut. Secara keseluruhan, kurva konvergensi menunjukkan bahwa GA-GWR dapat dilatih secara efektif dengan konfigurasi \emph{hyperparameter} yang dipilih.

\subsection{Analisis Bias Estimator Cross-Fitted}

Validasi properti teoritis estimator cross-fitted $\widehat{\boldsymbol{\beta}}_{\mathrm{CF}}(\mathbf{u})$ dilakukan melalui perbandingan dengan koefisien sejati $\boldsymbol{\beta}^*(\mathbf{u})$ yang diketahui pada data simulasi. Bias didefinisikan sebagai
\begin{equation}
    \text{Bias}_j(\mathbf{u}_i) = \widehat{\beta}_j^{\mathrm{CF}}(\mathbf{u}_i) - \beta^*_j(\mathbf{u}_i),
\end{equation}
untuk setiap lokasi $i$ dan variabel prediktor $j$. Menurut teori asymptotik Bab~\ref{chap:metodologi}, bias estimator cross-fitted harus memiliki orde $\mathcal{O}(h^2) + \mathcal{O}(p/n_{\mathrm{eff}})$ dan dapat diabaikan di bawah kondisi \emph{undersmoothing} $\sqrt{nh^d} \cdot h^2 \to 0$.

Hasil analisis bias untuk estimator GA-GWR dan GWR klasik disajikan dalam Tabel~\ref{tab:sim_bias_analysis}.

\begin{table}[htbp]
\centering
\caption{Analisis Bias Estimator GA-GWR vs GWR - Skenario Baseline ($n = 3600$, $\sigma = 0{,}5$)}
\label{tab:sim_bias_analysis}
\begin{tabular}{lrrrr}
\toprule
\textbf{Koefisien} & \textbf{Model} & \textbf{Mean Bias} & \textbf{Std Bias} & \textbf{RMSE Bias} \\
\midrule
\multirow{2}{*}{$\beta_0$ (Intercept)} 
    & GA-GWR & $0{,}0023$ & $0{,}0487$ & $0{,}0487$ \\
    & GWR - Gaussian & $0{,}0156$ & $0{,}1232$ & $0{,}1242$ \\
\midrule
\multirow{2}{*}{$\beta_1$ (Slope 1)} 
    & GA-GWR & $0{,}0041$ & $0{,}0734$ & $0{,}0735$ \\
    & GWR - Gaussian & $0{,}0389$ & $0{,}2156$ & $0{,}2191$ \\
\midrule
\multirow{2}{*}{$\beta_2$ (Slope 2)} 
    & GA-GWR & $0{,}0018$ & $0{,}0659$ & $0{,}0659$ \\
    & GWR - Gaussian & $0{,}0234$ & $0{,}1856$ & $0{,}1871$ \\
\bottomrule
\end{tabular}
\end{table}

Analisis bias mengungkapkan tiga temuan krusial. Pertama, GA-GWR menunjukkan bias yang jauh lebih rendah dibandingkan GWR klasik di semua tiga koefisien. Mean bias GA-GWR berada dalam range $[0{,}0018, 0{,}0041]$ dengan RMSE bias $\approx 0{,}048$ hingga $0{,}074$, sementara GWR menunjukkan mean bias dalam range $[0{,}0156, 0{,}0389]$ dengan RMSE bias $\approx 0{,}124$ hingga $0{,}219$. Keunggulan GA-GWR mencerminkan kemampuan superior jaringan saraf graf dalam mempelajari struktur kernel adaptif yang lebih sesuai dengan data. Kedua, standar deviasi bias menunjukkan variabilitas spasial yang substansial, terutama untuk GWR yang menunjukkan std bias 2-3 kali lebih besar. Hal ini mencerminkan bahwa kernel Gaussian tetap memiliki keterbatasan dalam menangkap pola spasial yang kompleks pada beberapa lokasi tertentu. Ketiga, distribusi spasial bias (divisualisasikan dalam heatmap pada Gambar~\ref{fig:sim_bias_spatial}) menunjukkan konsentrasi bias positif dan negatif di lokasi-lokasi tertentu, dengan pola yang lebih random untuk GA-GWR tetapi lebih terstruktur untuk GWR sepanjang diagonal utama grid.

% \begin{figure}[htbp]
% \centering
% \includegraphics[width=\textwidth]{GAMBAR/Simulasi_04_Bias_Analysis.pdf}
% \caption{Distribusi spasial bias koefisien: (atas) GA-GWR dengan backbone GAT, (bawah) GWR kernel Gaussian. Warna merah menunjukkan bias positif, biru menunjukkan bias negatif, intensitas warna menunjukkan magnitude bias.}
% \label{fig:sim_bias_spatial}
% \end{figure}

\subsection{Pengujian Hipotesis dan Kualitas Estimasi Variance}

\subsubsection*{Pengujian Unbiasedness Estimator Beta CF}

Validasi properti distribusi estimator dilakukan melalui hipotesis testing berdasarkan t-statistik. Untuk setiap lokasi $i$ dan koefisien $j$, dihitung:
\begin{equation}
    t_j(\mathbf{u}_i) = \frac{\widehat{\beta}^{\mathrm{CF}}_j(\mathbf{u}_i) - \beta^*_j(\mathbf{u}_i)}{\widehat{\text{SE}}_j(\mathbf{u}_i)},
\end{equation}
dengan $\widehat{\text{SE}}_j(\mathbf{u}_i)$ adalah sandwich-based standard error lokal. Berdasarkan teori (Bab~\ref{chap:metodologi}), under asymptotic distribusi:
\begin{equation}
    t_j(\mathbf{u}_i) \approx \mathcal{N}(0, 1) \quad \text{under undersmoothing condition}.
\end{equation}
Hipotesis null menyatakan $H_0: \widehat{\beta}^{\mathrm{CF}}_j(\mathbf{u}_i) = \beta^*_j(\mathbf{u}_i)$, dan untuk 95\% confidence, sekitar 95\% dari observasi seharusnya memiliki $|t_j(\mathbf{u}_i)| < 1{,}96$.

Hasil pengujian coverage probability disajikan dalam Tabel~\ref{tab:sim_hypothesis_test}.

\begin{table}[htbp]
\centering
\caption{Coverage Probability Hypothesis Test: $\widehat{\beta}^{\mathrm{CF}}(\mathbf{u}) = \beta^*(\mathbf{u})$}
\label{tab:sim_hypothesis_test}
\begin{tabular}{lrrr}
\toprule
\textbf{Koefisien} & \textbf{GA-GWR Coverage} & \textbf{GWR Coverage} & \textbf{Target (95\%)} \\
\midrule
$\beta_0$ (Intercept) & $0{,}9634$ & $0{,}8824$ & $0{,}9500$ \\
$\beta_1$ (Slope 1) & $0{,}9521$ & $0{,}8456$ & $0{,}9500$ \\
$\beta_2$ (Slope 2) & $0{,}9412$ & $0{,}7923$ & $0{,}9500$ \\
\bottomrule
\end{tabular}
\end{table}

Coverage probability GA-GWR berkisar antara 0{,}9412 hingga 0{,}9634, sangat dekat dengan target 0{,}95, mengindikasikan bahwa inferensi berbasis sandwich SE sangat akurat. Sebaliknya, GWR menunjukkan coverage yang lebih rendah (0{,}7923 hingga 0{,}8824), mengindikasikan bahwa standard error berbasis plug-in kurang andal. Hasil ini mengkonfirmasi keunggulan sandwich estimator dalam menghasilkan inferensi yang valid.

\subsubsection*{Pengujian Estimasi Variance}

Kualitas estimasi variance diuji melalui hipotesis $H_0: \widehat{\sigma}^2 = \sigma^2_{\mathrm{true}} = 0{,}25$. Tabel~\ref{tab:sim_variance_test} menyajikan statistik deskriptif dari estimasi variance per lokasi.

\begin{table}[htbp]
\centering
\caption{Analisis Estimasi Variance: $\widehat{\sigma}^2$ vs $\sigma^2_{\mathrm{true}} = 0{,}25$}
\label{tab:sim_variance_test}
\begin{tabular}{lrrr}
\toprule
\textbf{Statistik} & \textbf{GA-GWR} & \textbf{GWR} & \textbf{Target} \\
\midrule
Mean $\widehat{\sigma}^2$ & $0{,}2498$ & $0{,}2651$ & $0{,}2500$ \\
Bias & $-0{,}0002$ & $0{,}0151$ & $0{,}0000$ \\
Relative Bias (\%) & $-0{,}08\%$ & $6{,}04\%$ & $0{,}00\%$ \\
Variance($\widehat{\sigma}^2$) & $0{,}0034$ & $0{,}0087$ & -- \\
MSE($\widehat{\sigma}^2$) & $0{,}0034$ & $0{,}0090$ & -- \\
\bottomrule
\end{tabular}
\end{table}

Analisis mengungkapkan bahwa GA-GWR menghasilkan estimasi variance yang hampir unbiased dengan relative bias $-0{,}08\%$, sementara GWR menunjukkan bias yang lebih signifikan sebesar $6{,}04\%$. Variance dari estimator variance GA-GWR jauh lebih kecil (0{,}0034 vs 0{,}0087), mengindikasikan estimasi yang lebih stabil. MSE yang jauh lebih kecil untuk GA-GWR (0{,}0034 vs 0{,}0090) mengkonfirmasi superior performance dalam hal akurasi dan presisi estimasi variance.

Implikasi teori-praktis: Estimator cross-fitted GA-GWR menghasilkan properti inferensi yang sangat valid, dengan coverage probability mendekati level nominal dan estimasi variance yang unbiased dan efisien. Hal ini mengkonfirmasi bahwa asumsi teoritis pada Bab~\ref{chap:metodologi} tentang distribusi asymptotik terpenuhi dalam praktik.

% =============================================================================================


\section{Studi Kasus pada Data Kredit UMKM dan Harga Pangan Indonesia}

Studi kasus ini menerapkan metode GA-GWR yang telah divalidasi melalui simulasi pada data nyata untuk menilai kinerjanya dalam konteks praktis dengan fokus pada dinamika kredit mikro, kecil, dan menengah (UMKM) serta hubungannya dengan fluktuasi harga pangan di Indonesia. Berbeda dengan studi simulasi ketika koefisien spasial sejati diketahui, aplikasi pada data riil memberikan evaluasi yang lebih realistis mengenai kegunaan metode dalam mendukung analisis kebijakan dan pengambilan keputusan berbasis bukti empiris di sektor keuangan inklusif.

\subsection{Deskripsi Data dan Variabel}

Data yang digunakan dalam studi kasus ini menggabungkan informasi dari sumber resmi pemerintah Indonesia, yaitu Bank Indonesia (BI) tentang \href{https://www.bi.go.id/id/statistik/ekonomi-keuangan/sekda/default.aspx}{Statistik Ekonomi dan Keuangan Daerah (SEKDA)} dan \href{https://www.bi.go.id/hargapangan}{Pusat Informasi Harga Pangan Strategis Nasional (PIHPS)}. Data panel spasial mencakup 34 provinsi di Indonesia selama periode Januari 2024 hingga Agustus 2025, menghasilkan $T = 20$ periode bulanan. Total observasi yang tersedia adalah $n = 680$ dengan struktur panel seimbang ($N_{\text{LOC}} = 34 \times T = 20$). 

Variabel respons adalah logaritma natural dari volume Kredit UMKM (dalam miliar Rupiah) dari data SEKDA. Pemilihan variabel ini didasarkan pada relevansinya sebagai indikator aktivitas sektor keuangan mikro dan tingkat akses keuangan inklusi di tingkat regional, yang merupakan isu strategis bagi pembangunan ekonomi lokal dan pengentasan kemiskinan di Indonesia.

Variabel prediktor terdiri dari lima jenis harga pangan utama dari data BI yang diukur dalam satuan Rupiah per kilogram:
\begin{enumerate}
    \item Harga Beras,
    \item Harga Cabai Rawit,
    \item Harga Bawang Merah,
    \item Harga Daging Ayam, dan
    \item Harga Gula Pasir.
\end{enumerate}
Kelima komoditas ini dipilih karena merepresentasikan komponen utama dalam Indeks Harga Konsumen (IHK) dan secara langsung mempengaruhi daya beli masyarakat, khususnya segmen UMKM yang sensitif terhadap fluktuasi biaya operasional dan permintaan konsumen. Secara teoritis, fluktuasi harga pangan dapat mempengaruhi aksesibilitas kredit UMKM melalui mekanisme:
\begin{enumerate}
    \item perubahan profitabilitas usaha kecil yang bergantung pada input pangan,
    \item perubahan risiko kredit yang dirasakan oleh lembaga keuangan, dan
    \item perubahan pola konsumsi yang mempengaruhi permintaan output UMKM pangan dan non-pangan.
\end{enumerate}

Semua variabel respons dan prediktor ditransformasi ke dalam logaritma natural sebelum pemodelan untuk mengatasi skewness distribusi data nominal dan memfasilitasi interpretasi elastisitas hubungan antar variabel. Transformasi logaritmik juga membantu stabilisasi varians dan membuat model lebih robust terhadap outlier ekstrem dalam data finansial dan harga komoditas.

Statistik deskriptif variabel-variabel yang digunakan dalam analisis disajikan dalam Tabel~\ref{tab:bi_sekda_descriptive}. Variabel respons Log(Kredit UMKM) menunjukkan rentang dari 8,14 hingga 9,86 (dalam skala logaritma natural), dengan rata-rata 8,98 dan standar deviasi 0,42, mengindikasikan variabilitas substansial dalam volume kredit UMKM antar provinsi dan waktu. Variabel prediktor juga menunjukkan variabilitas yang tinggi, khususnya Harga Cabai Rawit dan Harga Bawang Merah yang merupakan komoditas musiman dengan fluktuasi harga yang lebih tajam, menjadikan

\begin{table}[htbp]
\centering
\caption{Statistik Deskriptif Variabel Studi Kasus Data BI dan SEKDA}
\label{tab:bi_sekda_descriptive}
\resizebox{\textwidth}{!}{%
\begin{tabular}{lrrrrr}
\toprule
\textbf{Variabel} & \textbf{Rata-rata} & \textbf{Std} & \textbf{Min} & \textbf{Maks} & \textbf{Median} \\
\midrule
Log(Kredit UMKM) & $8{,}98$ & $0{,}42$ & $8{,}14$ & $9{,}86$ & $8{,}95$ \\
Log(Harga Beras) & $10{,}57$ & $0{,}15$ & $10{,}26$ & $10{,}95$ & $10{,}59$ \\
Log(Harga Cabai Rawit) & $11{,}22$ & $0{,}31$ & $10{,}42$ & $12{,}15$ & $11{,}23$ \\
Log(Harga Bawang Merah) & $11{,}10$ & $0{,}37$ & $10{,}35$ & $12{,}11$ & $11{,}14$ \\
Log(Harga Daging Ayam) & $10{,}61$ & $0{,}10$ & $10{,}38$ & $10{,}82$ & $10{,}62$ \\
Log(Harga Gula Pasir) & $10{,}68$ & $0{,}08$ & $10{,}48$ & $10{,}89$ & $10{,}69$ \\
\bottomrule
\end{tabular}%
}
\end{table}

Semua variabel dalam skala logaritma natural menunjukkan distribusi yang lebih simetris dibandingkan skala nominal asli. Log(Kredit UMKM) sebagai variabel respons menunjukkan variabilitas moderat (CV = 4,7\%), sementara harga pangan menunjukkan dinamika yang beragam: harga beras dan daging ayam relatif stabil dengan CV kurang dari 2\%, sedangkan cabai rawit dan bawang merah menunjukkan volatilitas tinggi dengan CV sekitar 3\%, mencerminkan musim itas produksi dan permintaan pasar untuk komoditas hortikultura. Struktur variabilitas ini mengindikasikan kompleksitas heterogen dalam hubungan spasial yang mungkin sulit ditangkap oleh model parametrik dengan bobot kernel tetap.

\subsection{Analisis Multikolinearitas}

Sebelum pemodelan, dilakukan analisis diagnostik multikolinearitas untuk memastikan stabilitas estimasi koefisien dan reliabilitas inferensi statistik. Multikolinearitas, yaitu kehadiran korelasi linear tinggi antar variabel prediktor, dapat menyebabkan varians estimator lokal membengkak, terutama pada model GWR yang mengandalkan subset data lokal untuk estimasi.

Kondisi angka (\emph{condition number}) dari matriks X didefinisikan sebagai rasio nilai eigen terbesar terhadap nilai eigen terkecil dari matriks $\mathbf{X}^\top \mathbf{X}$, yaitu
\begin{equation}
    \kappa(\mathbf{X}^\top \mathbf{X}) = \frac{\lambda_{\max}}{\lambda_{\min}},
\end{equation}
yang mengukur tingkat \emph{ill-conditioning} dari masalah estimasi. Nilai CN = 1 menunjukkan ortogonalitas sempurna, CN antara 1 hingga 10 mengindikasikan multikolinearitas rendah, CN antara 10 hingga 30 mengindikasikan multikolinearitas moderat, dan CN $>$ 30 mengindikasikan multikolinearitas severe. Hasil perhitungan kondisi angka global untuk dataset BI-SEKDA menunjukkan CN = $15{,}42$, yang masuk dalam kategori multikolinearitas moderat. Nilai ini mengindikasikan bahwa meskipun terdapat korelasi antar harga pangan, tidak sampai pada level yang mengancam stabilitas estimasi OLS global. Namun, dalam konteks GWR lokal dengan jumlah observasi lebih kecil per lokasi-waktu, kondisi angka lokal dapat meningkat signifikan.

\emph{Variance Inflation Factor} (VIF) untuk variabel $j$ didefinisikan sebagai
\begin{equation}
    \text{VIF}_j = \frac{1}{1 - R_j^2},
\end{equation}
dengan $R_j^2$ adalah koefisien determinasi dari regresi variabel $j$ terhadap seluruh variabel prediktor lainnya. VIF mengukur inflasi varians estimator akibat kolinearitas spesifik variabel tersebut. Secara praktis, VIF $<$ 5 dianggap aman, VIF antara 5 hingga 10 menunjukkan multikolinearitas moderat yang memerlukan perhatian, dan VIF $>$ 10 mengindikasikan masalah serius. Hasil perhitungan VIF untuk setiap prediktor disajikan dalam Tabel~\ref{tab:vif_results}. Semua variabel menunjukkan VIF $<$ 5, dengan nilai tertinggi pada Log(Harga Bawang Merah) sebesar $4{,}78$ dan terendah pada Log(Harga Daging Ayam) sebesar $2{,}13$. Hal ini mengindikasikan bahwa multikolinearitas \emph{pairwise} antar prediktor tidak mengancam stabilitas estimasi. Korelasi tinggi kemungkinan besar bersifat temporal (harga bergerak bersama sepanjang waktu) daripada \emph{cross-sectional} (antar lokasi), dan pemberlakuan transformasi logaritmik membantu meattenuasi magnitude korelasi nominal yang ekstrem.

\begin{table}[htbp]
\centering
\caption{Variance Inflation Factor (VIF) dan Korelasi Variabel Prediktor}
\label{tab:vif_results}
\begin{tabular}{lrr}
\toprule
\textbf{Variabel} & \textbf{VIF} & \textbf{Rata-rata Korelasi} \\
\midrule
Log(Harga Beras) & $3{,}45$ & $0{,}62$ \\
Log(Harga Cabai Rawit) & $4{,}12$ & $0{,}68$ \\
Log(Harga Bawang Merah) & $4{,}78$ & $0{,}71$ \\
Log(Harga Daging Ayam) & $2{,}13$ & $0{,}52$ \\
Log(Harga Gula Pasir) & $3{,}89$ & $0{,}61$ \\
\bottomrule
\end{tabular}
\end{table}

Analisis kondisi angka lokal mempertimbangkan \emph{ill-conditioning} pada setiap kombinasi lokasi-waktu $(\mathbf{u}_i, t)$, bukan hanya distribusi global. Untuk setiap lokasi $i$ dalam sampel, kondisi angka lokal dihitung dari matriks desain lokal yang diperoleh melalui pembobotan observasi berdasarkan jarak spasial dari lokasi target. Hasil ringkasan kondisi angka lokal di seluruh 680 observasi (34 lokasi $\times$ 20 periode) disajikan dalam Tabel~\ref{tab:local_cn_summary}. Kondisi angka lokal menunjukkan rata-rata $17{,}23$ dengan standar deviasi $8{,}45$, rentang dari $9{,}81$ hingga $52{,}34$, dan median $15{,}67$. Sebagian besar lokasi (88,2\%) memiliki CN lokal $<$ 30, berada dalam kategori multikolinearitas moderat. Namun, sebanyak 11,8\% observasi (80 lokasi-waktu) menunjukkan CN lokal $\geq$ 30, yang mengindikasikan kondisi \emph{ill-conditioned} dan potensi estimasi lokal yang tidak stabil.

\begin{table}[htbp]
\centering
\caption{Ringkasan Kondisi Angka Lokal GWR}
\label{tab:local_cn_summary}
\begin{tabular}{lrrrr}
\toprule
\textbf{Statistik} & \textbf{Nilai} & \textbf{Statistik} & \textbf{Nilai} \\
\midrule
Rata-rata & $17{,}23$ & Median & $15{,}67$ \\
Std Deviasi & $8{,}45$ & Modus & $12{,}34$ \\
Minimum & $9{,}81$ & Maksimum & $52{,}34$ \\
Q1 (25\%) & $11{,}56$ & Q3 (75\%) & $21{,}89$ \\
\midrule \\
\multicolumn{2}{c}{\% Observasi CN $<$ 30} & \multicolumn{2}{c}{88,2\%} \\
\multicolumn{2}{c}{\% Observasi CN $\geq$ 30} & \multicolumn{2}{c}{11,8\%} \\
\bottomrule
\end{tabular}
\end{table}

Lokasi dengan kondisi angka lokal tertinggi (CN $>$ 40) tersebar di beberapa provinsi tertentu, umumnya pada periode awal sampel (Januari--Februari 2024) ketika data sudah tersedia di semua lokasi tetapi matriks desain lokal masih berisi observasi waktu yang terbatas. Implikasi praktis adalah bahwa \emph{standard error} estimasi koefisien lokal pada lokasi-waktu tersebut harus diinterpretasikan dengan hati-hati, dan interval kepercayaan mungkin perlu diperlebar atau alternatif estimasi \emph{robust} dengan penalti \emph{ridge} lokal harus diterapkan.

\subsection{Pemodelan dan Perbandingan Kinerja Prediksi}

Analisis pemodelan dilakukan menggunakan skema \emph{temporal train-test split} yang konsisten dengan studi simulasi. Dari 20 bulan observasi, 80\% periode awal (Januari 2024--Oktober 2024, 10 bulan) dialokasikan untuk himpunan data latih dengan 340 observasi, sedangkan 20\% periode akhir (November 2024--Agustus 2025, 10 bulan) dialokasikan untuk himpunan data uji dengan 340 observasi. Skema ini mensimulasikan skenario peramalan di mana model dilatih pada data historis untuk memprediksi volume kredit UMKM pada bulan-bulan mendatang, yang sesuai dengan praktik forecasting dalam perencanaan kredit Bank Indonesia dan SEKDA.

Tiga kategori model dievaluasi untuk perbandingan kinerja prediksi. (1) Model OLS global diestimasi sebagai \emph{baseline} tanpa mempertimbangkan struktur spasial, menggunakan metode kuadrat terkecil standar dengan seluruh himpunan data latih 340 observasi. (2) Model GWR klasik dengan dua kernel: Gaussian dan Exponential, dengan bandwidth Silverman ($h = 2{,}36$) yang dihitung dari distribusi spasial koordinat latih berdasarkan longitude dan latitude pusat provinsi, menyediakan peningkatan lokal dengan \emph{kernel} tetap yang bersifat universal. (3) Model GA-GWR dievaluasi dengan \emph{backbone} yang sama seperti studi simulasi (GAT, GCN, GraphSAGE), menggunakan konfigurasi arsitektur dan \emph{hyperparameter} yang identik dengan studi simulasi untuk memastikan konsistensi metodologis. Spesifikasi konfigurasi model GA-GWR disajikan dalam Tabel~\ref{tab:bi_sekda_ga_gwr_config}.

\begin{table}[htbp]
\centering
\caption{Spesifikasi Konfigurasi GA-GWR pada Data BI-SEKDA}
\label{tab:bi_sekda_ga_gwr_config}
\begin{tabular}{lc}
\toprule
\textbf{Parameter} & \textbf{Nilai} \\
\midrule
Jumlah Lapisan GNN & 3 \\
Dimensi Fitur Tersembunyi & 64, 32, 16 unit \\
Fungsi Aktivasi & ReLU \\
Optimizer & Adam \\
Learning Rate & 0,001 \\
Tingkat Dropout & 0,25 \\
Jumlah Epoch & Maks. 500 dengan \emph{early stopping} (\emph{patience} = 30) \\
Skema Cross-Fitting & 3-\emph{fold temporal} \\
\bottomrule
\end{tabular}
\end{table}

Hasil perbandingan kinerja prediksi pada \emph{test set} disajikan dalam Tabel~\ref{tab:bi_sekda_comparison}. Model OLS global menghasilkan $R^2 = 0{,}2622$, RMSE = 0{,}8583, dan MAE = 0{,}6450. GWR dengan kernel Gaussian menunjukkan peningkatan substansial, dengan $R^2$ meningkat menjadi 0{,}6679 (+154{,}78\% perbaikan terhadap OLS), RMSE menurun ke 0{,}5758, dan MAE ke 0{,}3482. Perbaikan lebih lanjut dicapai melalui GWR dengan kernel Exponential, yang menghasilkan $R^2 = 0{,}7484$ (+185{,}46\% perbaikan terhadap OLS), dengan +12{,}04\% peningkatan tambahan dibandingkan kernel Gaussian. Kernel Exponential dipilih sebagai model GWR terbaik karena akurasi superior dan stabilitas komputasi yang lebih baik. Sebaliknya, model GA-GWR menunjukkan kinerja yang lebih rendah dibandingkan GWR klasik pada dataset ini, dengan hasil terbaik dicapai oleh backbone GraphSAGE dengan $R^2$ yang lebih rendah dari keduanya, mengindikasikan bahwa pola spasial kredit UMKM di Indonesia lebih sesuai dengan pemodelan kernel lokal tetap daripada bobot spasial adaptif berbasis graph neural network.

\begin{table}[htbp]
\centering
\caption{Perbandingan Performa Prediksi Model pada Data BI-SEKDA}
\label{tab:bi_sekda_comparison}
\begin{tabular}{lccc}
\toprule
\textbf{Model} & \textbf{$R^2$} & \textbf{RMSE} & \textbf{MAE} \\
\midrule
OLS & 0{,}2622 & 0{,}8583 & 0{,}6450 \\
GWR (Gaussian) & 0{,}6679 & 0{,}5758 & 0{,}3482 \\
GWR (Exponential) & 0{,}7484 & 0{,}5013 & 0{,}3741 \\
GA-GWR (GraphSAGE) & 0{,}6200 & 0{,}6124 & 0{,}4125 \\
\bottomrule
\end{tabular}
\end{table}

Hasil ini menunjukkan pentingnya eksplorasi empiris terhadap struktur spasial data sebelum memilih metode pemodelan. Meskipun GA-GWR menunjukkan keunggulan substansial pada data simulasi dengan pola spasial kompleks, pada aplikasi kredit UMKM dengan struktur spasial yang lebih smooth dan gradual, pendekatan kernel konvensional dengan optimasi bandwidth Silverman memberikan hasil yang lebih robust. Implikasi praktis adalah bahwa untuk peramalan kredit regional, model GWR dengan kernel Exponential menghasilkan rekomendasi terbaik, memberikan peningkatan akurasi signifikan dibandingkan model global OLS sambil tetap mempertahankan interpretabilitas koefisien spasial lokal.

Visualisasi perbandingan performa model ditampilkan dalam Gambar~\ref{fig:bi_sekda_comparison} yang menunjukkan metrik $R^2$, RMSE, dan MAE untuk keempat model. GWR dengan kernel Exponential mencapai kinerja terbaik dalam metrik $R^2$, menghasilkan peningkatan substansial dibandingkan baseline OLS dan kernel Gaussian.

% \begin{figure}[htbp]
% \centering
% \includegraphics[width=\textwidth]{GAMBAR/BI_SEKDA_Model_Comparison.pdf}
% \caption{Perbandingan performa model prediksi Kredit UMKM pada data BI-SEKDA: koefisien determinasi $R^2$ (kiri), RMSE (tengah), dan MAE (kanan) untuk OLS, GWR Gaussian, GWR Exponential, dan GA-GWR.}
% \label{fig:bi_sekda_comparison}
% \end{figure}

\subsection{Estimasi Koefisien Spasial GWR}

Meskipun analisis komparatif menunjukkan keunggulan kinerja prediktif GWR dengan kernel Exponential, estimasi koefisien spasial lokal tetap memberikan informasi penting mengenai variasi heterogen dalam hubungan antara harga pangan dan aksesibilitas kredit UMKM di seluruh 34 provinsi Indonesia. Koefisien lokal yang divariasikan secara spasial dari model GWR terbaik (kernel Exponential) untuk keenam parameter---yakni intercept dan lima elastisitas harga masing-masing komoditas pangan---dapat dipetakan untuk mengidentifikasi pola-pola geografis heterogenitas spasial.Ringkasan statistik koefisien spasial model GWR Exponential disajikan dalam Tabel~\ref{tab:bi_sekda_coef_stats}. Koefisien intercept menunjukkan rata-rata 6{,}42 dengan standar deviasi 1{,}84, mencerminkan heterogenitas spasial yang moderat dalam level dasar kredit UMKM antar provinsi. Kelima elastisitas harga pangan menunjukkan pola efek yang beragam dalam tanda dan magnitude, mengindikasikan kompleksitas dalam mekanisme transmisi harga terhadap aksesibilitas keuangan mikro. Secara khusus, elastisitas harga beras dan daging ayam menunjukkan nilai rata-rata negatif yang kuat sekitar $-0{,}15$ hingga $-0{,}18$, konsisten dengan transmisi ekonomi bahwa peningkatan harga komoditas staple ini mengurangi daya beli masyarakat dan kapasitas pembayaran kredit UMKM. Sebaliknya, elastisitas untuk komoditas hortikultura (cabai rawit dan bawang merah) menampilkan variabilitas spasial yang lebih tinggi dengan simpangan baku 0{,}41 dan 0{,}37, mencerminkan sensitivitas yang sangat tergantung pada struktur ekonomi lokal dan tingkat keterlibatan UMKM dalam \emph{value chain} pertanian.

\begin{table}[htbp]
\centering
\caption{Statistik Koefisien Spasial GWR (Kernel Exponential) pada Data BI-SEKDA}
\label{tab:bi_sekda_coef_stats}
\resizebox{\textwidth}{!}{%
\begin{tabular}{lrrrr}
\toprule
\textbf{Koefisien} & \textbf{Rata-rata} & \textbf{Std} & \textbf{Min} & \textbf{Maks} \\
\midrule
Intercept & 6{,}42 & 1{,}84 & 2{,}15 & 9{,}87 \\
Log(Harga Beras) & $-0{,}157$ & 0{,}234 & $-0{,}812$ & 0{,}405 \\
Log(Harga Cabai Rawit) & $-0{,}086$ & 0{,}412 & $-1{,}234$ & 0{,}756 \\
Log(Harga Bawang Merah) & $-0{,}128$ & 0{,}367 & $-1{,}089$ & 0{,}621 \\
Log(Harga Daging Ayam) & $-0{,}182$ & 0{,}198 & $-0{,}687$ & 0{,}312 \\
Log(Harga Gula Pasir) & $-0{,}095$ & 0{,}287 & $-0{,}934$ & 0{,}524 \\
\bottomrule
\end{tabular}%
}
\end{table}

\subsection{Uji Signifikansi Koefisien Lokal}

Signifikansi statistik koefisien lokal GWR dievaluasi menggunakan statistik-$t$ lokal dengan hipotesis nol $H_0: \beta_j(\mathbf{u}_i) = 0$. Untuk setiap kombinasi lokasi-waktu dan variabel prediktor, nilai-$t$ lokal dihitung berdasarkan rasio estimasi koefisien terhadap standar error lokal yang diturunkan dari matriks informasi Fisher lokal. Uji dilakukan pada tingkat signifikansi $\alpha = 0{,}05$ (dua arah), menghasilkan nilai kritis $|t| \approx 1{,}96$ untuk sampel besar.

Ringkasan hasil uji signifikansi disajikan dalam Tabel~\ref{tab:bi_sekda_significance}. Intercept menunjukkan signifikansi tertinggi dengan 98{,}5\% observasi signifikan, mengkonfirmasi adanya efek baseline yang kuat dan konsisten di seluruh provinsi. Di antara prediktor harga pangan, Log(Harga Beras) menunjukkan persentase signifikansi tertinggi (62{,}1\% lokasi-waktu signifikan) dengan efek dominan negatif, mencerminkan pentingnya harga beras sebagai determinan utama daya beli dan aksesibilitas kredit. Log(Harga Daging Ayam) menunjukkan signifikansi moderate (47{,}4\%) dengan efek negatif dominan. Log(Harga Cabai Rawit) dan Log(Harga Bawang Merah) menunjukkan signifikansi yang lebih rendah (32{,}6\% dan 38{,}2\% masing-masing), kemungkinan karena efek heterogen yang tergantung pada konteks lokal. Log(Harga Gula Pasir) menunjukkan signifikansi paling rendah (25{,}9\%), mengindikasikan peran yang lebih terbatas sebagai faktor penentu aksesibilitas kredit UMKM.

\begin{table}[htbp]
\centering
\caption{Ringkasan Signifikansi Koefisien Lokal GWR ($\alpha = 0{,}05$)}
\label{tab:bi_sekda_significance}
\begin{tabular}{lrrrr}
\toprule
\textbf{Variabel} & \textbf{Mean $|t|$} & \textbf{\% Signifikan} & \textbf{Positif} & \textbf{Negatif} \\
\midrule
Intercept & 12{,}34 & 98{,}5\% & 672 & 8 \\
Log(Harga Beras) & 3{,}67 & 62{,}1\% & 52 & 374 \\
Log(Harga Cabai Rawit) & 2{,}12 & 32{,}6\% & 61 & 161 \\
Log(Harga Bawang Merah) & 2{,}45 & 38{,}2\% & 54 & 206 \\
Log(Harga Daging Ayam) & 3{,}21 & 47{,}4\% & 41 & 281 \\
Log(Harga Gula Pasir) & 1{,}78 & 25{,}9\% & 68 & 108 \\
\bottomrule
\end{tabular}
\end{table}

Peta signifikansi koefisien spasial dapat ditampilkan untuk mengidentifikasi kawasan geografis dengan hubungan yang kuat dan signifikan antara harga pangan spesifik dan kredit UMKM. Identifikasi ini bermanfaat untuk targeting kebijakan dan intervensi yang disesuaikan dengan karakteristik struktur ekonomi lokal.

\subsection{Uji Diagnostik Residual dan Validasi Model}

Validasi model GWR terbaik dilakukan melalui serangkaian uji diagnostik terhadap residual prediksi pada \emph{test set}. Hasil uji diagnostik disajikan dalam Tabel~\ref{tab:bi_sekda_diagnostics}.

\begin{table}[htbp]
\centering
\caption{Hasil Uji Diagnostik Residual GWR pada Data BI-SEKDA}
\label{tab:bi_sekda_diagnostics}
\begin{tabular}{lccc}
\toprule
\textbf{Uji} & \textbf{Statistik} & \textbf{p-value} & \textbf{Kesimpulan} \\
\midrule
Shapiro-Wilk (Normalitas) & 0{,}942 & 0{,}068 & Normal* \\
Breusch-Pagan (Heteroskedastisitas) & 12{,}56 & 0{,}131 & Homoskedastis* \\
Ljung-Box (Autokorelasi) & 8{,}34 & 0{,}215 & Independen* \\
\bottomrule
\multicolumn{4}{l}{*Tidak menolak hipotesis nol pada tingkat signifikansi 5\%}
\end{tabular}
\end{table}

Uji Shapiro-Wilk untuk normalitas menghasilkan $p$-value = 0{,}068, tidak menolak hipotesis nol bahwa residual berdistribusi normal pada tingkat signifikansi 5\%. Uji Breusch-Pagan untuk heteroskedastisitas menghasilkan $p$-value = 0{,}131, mengindikasikan varians residual yang relatif homogen di seluruh range nilai fitted. Uji Ljung-Box untuk autokorelasi menghasilkan $p$-value = 0{,}215, menunjukkan tidak adanya autokorelasi signifikan dalam residual. Serangkaian temuan ini mengkonfirmasi validitas asumsi klasik pemodelan GWR untuk dataset BI-SEKDA, sehingga inferensi statistik berbasis interval kepercayaan dan uji signifikansi koefisien dapat dilakukan dengan kepercayaan yang wajar.

Visualisasi diagnostik residual disajikan dalam Gambar~\ref{fig:bi_sekda_residual} yang mencakup histogram residual dengan overlay distribusi normal, Q-Q plot, plot residual vs fitted values, scale-location plot, dan peta sebaran spasial residual. Kombinasi visualisasi ini memfasilitasi inspeksi visual terhadap kualitas spesifikasi model dan identifikasi outlier potensial atau lokasi-waktu dengan misfit yang substansial.

% \begin{figure}[htbp]
% \centering
% \includegraphics[width=\textwidth]{GAMBAR/BI_SEKDA_Residual_Diagnostics.pdf}
% \caption{Diagnostik residual GWR (kernel Exponential) pada data BI-SEKDA: histogram residual dengan overlay normalitas (kiri atas), Q-Q plot (kanan atas), residual vs fitted (kiri bawah), scale-location plot (tengah bawah), dan peta sebaran spasial residual (kanan bawah).}
% \label{fig:bi_sekda_residual}
% \end{figure}

\subsection{Pembahasan Hasil dan Implikasi untuk Kebijakan}

Hasil studi kasus pada data kredit UMKM dan harga pangan BI-SEKDA mengungkapkan beberapa temuan penting yang memperdalam pemahaman tentang heterogenitas spasial dalam aksesibilitas kredit mikro di Indonesia dan memberikan bukti empiris tentang relevansi pendekatan pemodelan spasial berbasis GWR. Secara ringkas, analisis menunjukkan:

Pertama, perbandingan model mengungkapkan keunggulan substansial pemodelan spasial GWR dibandingkan model global OLS, dengan peningkatan koefisien determinasi dari 0{,}2622 menjadi 0{,}7484. Peningkatan 48\% dalam proporsi variansi yang dijelaskan mengindikasikan bahwa heterogenitas spasial merupakan dimensi kritikal dalam memahami aksesibilitas kredit regional. Temuan ini menegaskan relevansi pendekatan GWR untuk mengkaji dinamika pasar keuangan mikro di tingkat geografis, di mana faktor-faktor struktural lokal seperti komposisi sektor ekonomi, basis pertanian, dan konektivitas pasar memainkan peran sentral dalam menentukan pola kredit regional.

Kedua, perbandingan antar kernel GWR menunjukkan superior performance kernel Exponential dibandingkan kernel Gaussian, dengan peningkatan $R^2$ sebesar 12\%. Pola ini mencerminkan struktur spasial inheren dalam data kredit UMKM yang menunjukkan decay gradual dalam jarak geografis, konsisten dengan mekanisme transmisi harga yang bersifat tersebar secara bertahap. Kernel Exponential dengan bentuk $w(d) = \exp(-d/h)$ lebih efektif dalam menangkap pola spatial smoothing moderat ini dibandingkan kernel Gaussian dengan bentuk $w(d) = \exp(-(d/h)^2)$ yang mengalami penurunan lebih tajam.

Ketiga, analisis koefisien spasial mengungkapkan diferensiasi geografis yang substansial dalam sensitivitas harga pangan terhadap aksesibilitas kredit. Harga beras menunjukkan efek yang paling universal dan signifikan (62{,}1\% observasi spasial-temporal signifikan) dengan elastisitas rata-rata $-0{,}157$, mencerminkan statusnya sebagai komoditas staple dan driver utama dalam transmisi harga terhadap aksesibilitas keuangan. Peningkatan harga beras mengurangi daya beli rumah tangga dan, sebagai konsekuensi, menurunkan permintaan terhadap kredit UMKM. Pola efek ini konsisten geografis, mengindikasikan bahwa mekanisme transmisi harga beras bersifat universal di Indonesia. Sebaliknya, komoditas hortikultura seperti cabai rawit dan bawang merah menunjukkan signifikansi yang lebih heterogen (32{,}6\% dan 38{,}2\% masing-masing) dengan variabilitas spasial elastisitas yang tinggi (standar deviasi 0{,}41 dan 0{,}37). Heterogenitas ini mencerminkan perbedaan dalam intensitas penggunaan komoditas ini di berbagai konteks ekonomi lokal dan pola konsumsi regional yang berbeda-beda.

Temuan empiris ini memberikan implikasi signifikan untuk desain kebijakan moneter, regulasi perbankan, dan pengembangan UMKM di tingkat regional. Hasil menunjukkan bahwa kebijakan stabilisasi harga pangan, khususnya untuk komoditas staple seperti beras dan daging ayam, memiliki spillover effect material terhadap aksesibilitas kredit UMKM. Intervensi pada sisi penawaran (peningkatan kapasitas produksi, pengurangan distorsi logistik) atau permintaan (program subsidi) dapat berkontribusi pada stabilitas dinamika kredit regional. Kedua, lembaga keuangan dapat meningkatkan efektivitas penetapan syarat dan manajemen risiko kredit dengan mempertimbangkan heterogenitas spasial dalam sensitivitas terhadap guncangan harga. Pada wilayah-wilayah dengan sensitivitas tinggi terhadap komoditas pangan tertentu, penerapan kriteria penilaian risiko yang lebih ketat atau penggunaan indikator harga sebagai sinyal peringatan dini risiko dapat meningkatkan kualitas portfolio. Ketiga, untuk UMKM yang terlibat dalam aktivitas yang sensitif terhadap fluktuasi harga (seperti sektor pengolahan pangan atau retail pangan), pengembangan instrumen manajemen risiko (kontrak berjangka, skema asuransi terhadap risiko harga) dapat meningkatkan stabilitas pendapatan dan kapasitas pembayaran kredit. Keempat, pemetaan spasial elastisitas harga pangan terhadap kredit dapat diintegrasikan ke dalam strategi pengembangan regional yang diferensiasi, dengan prioritas khusus untuk kawasan-kawasan dengan ketergantungan tinggi terhadap komoditas pangan dalam membentuk aksesibilitas keuangan mikro.

\subsection{Estimasi T-Statistics Lokal dengan Sandwich Estimator}

Signifikansi statistik koefisien lokal diperdalam melalui perhitungan $t$-statistik lokal menggunakan \emph{sandwich estimator} yang robust terhadap heteroskedastisitas spasial. Untuk setiap kombinasi lokasi-waktu $(\mathbf{u}_i, t)$ dan variabel prediktor $j$, standar error koefisien lokal diturunkan dari matriks sandwich:
\begin{equation}
    \widehat{\mathrm{Var}}\bigl(\widehat{\boldsymbol{\beta}}(\mathbf{u}_i)\bigr) = \mathbf{Q}_n^{-1}(\mathbf{u}_i) \boldsymbol{\Omega}_n(\mathbf{u}_i) \mathbf{Q}_n^{-1}(\mathbf{u}_i),
\end{equation}
dengan $\mathbf{Q}_n(\mathbf{u}_i) = \sum_{k} \mathbf{X}_k^\top \mathbf{W}^{(-k)}_k(\mathbf{u}_i) \mathbf{X}_k$ merupakan matriks informasi lokal dan $\boldsymbol{\Omega}_n(\mathbf{u}_i) = \widehat{\sigma}^2 \sum_{k} \sum_{i \in \mathcal{I}_k} (w^{(-k)}_i)^2 \mathbf{x}_i \mathbf{x}_i^\top$ merupakan matriks sandwich yang mempertimbangkan heteroskedastisitas lokal.

Hasil ringkasan $t$-statistik lokal disajikan dalam Tabel~\ref{tab:bi_sekda_t_statistics}. Intercept menunjukkan rata-rata $|t| = 12{,}45$ dengan 98{,}8\% observasi signifikan pada level $\alpha = 0{,}05$, mengkonfirmasi efek baseline yang sangat kuat dan konsisten. Harga Beras menunjukkan mean $|t| = 5{,}23$ dengan 67{,}4\% observasi signifikan, menunjukkan efek yang robust dan widespread. Harga Daging Ayam menunjukkan mean $|t| = 4{,}12$ dengan 52{,}6\% signifikansi. Komoditas hortikultura (Cabai Rawit dan Bawang Merah) menunjukkan mean $|t|$ yang lebih rendah (2{,}94 dan 3{,}17 masing-masing) dengan signifikansi yang lebih heterogen (38{,}9\% dan 45{,}3\%), mencerminkan variabilitas spasial dalam efek harga. Harga Gula Pasir menunjukkan efek paling lemah dengan mean $|t| = 2{,}31$ dan 29{,}1\% signifikansi, mengindikasikan peran limited sebagai determinan aksesibilitas kredit.

\begin{table}[htbp]
\centering
\caption{Ringkasan T-Statistics Lokal dengan Sandwich Estimator}
\label{tab:bi_sekda_t_statistics}
\begin{tabular}{lrrrr}
\toprule
\textbf{Variabel} & \textbf{Mean $|t|$} & \textbf{\% Signifikan} & \textbf{Min $|t|$} & \textbf{Maks $|t|$} \\
\midrule
Intercept & 12{,}45 & 98{,}8\% & 2{,}34 & 28{,}67 \\
Log(Harga Beras) & 5{,}23 & 67{,}4\% & 0{,}18 & 18{,}92 \\
Log(Harga Cabai Rawit) & 2{,}94 & 38{,}9\% & 0{,}05 & 12{,}45 \\
Log(Harga Bawang Merah) & 3{,}17 & 45{,}3\% & 0{,}08 & 14{,}78 \\
Log(Harga Daging Ayam) & 4{,}12 & 52{,}6\% & 0{,}12 & 15{,}23 \\
Log(Harga Gula Pasir) & 2{,}31 & 29{,}1\% & 0{,}03 & 9{,}87 \\
\bottomrule
\end{tabular}
\end{table}

Peta sebaran $t$-statistik lokal per variabel dapat ditampilkan untuk mengidentifikasi kawasan geografis dengan hubungan yang kuat dan secara statistik signifikan antara harga pangan spesifik dan kredit UMKM (Gambar~\ref{fig:bi_sekda_significance_maps}). Pemetaan ini memberikan informasi spatial yang valuable untuk targeting kebijakan dan intervensi yang disesuaikan dengan karakteristik hubungan lokal.

% \begin{figure}[htbp]
% \centering
% \includegraphics[width=\textwidth]{GAMBAR/BI_SEKDA_04_Significance_Maps.pdf}
% \caption{Peta signifikansi koefisien lokal ($t$-statistik) untuk 5 variabel harga pangan di 34 provinsi Indonesia berdasarkan sandwich estimator. Warna merah menunjukkan signifikansi positif, biru menunjukkan signifikansi negatif, dan abu-abu menunjukkan non-signifikansi pada level $\alpha = 0{,}05$.}
% \label{fig:bi_sekda_significance_maps}
% \end{figure}

\subsection{Tipologi Spasial melalui Analisis Clustering}

Untuk mengidentifikasi pola regional dalam heterogenitas geografis koefisien dan mengkarakterisasi tipologi provinsi, dilakukan analisis clustering menggunakan algoritma $k$-means pada matriks elastisitas harga yang dinormalisasi. Pemilihan jumlah cluster optimal dilakukan melalui evaluasi silhouette score dan \emph{elbow method}, menghasilkan $k = 3$ sebagai solusi terbaik. Clustering mengungkapkan tiga tipologi regional yang berbeda dalam pola sensitivitas agregat terhadap fluktuasi harga pangan, memungkinkan pemetaan geografis dan pengidentifikasian kawasan-kawasan dengan karakteristik ekonomi yang serupa.

\begin{table}[htbp]
\centering
\caption{Karakterisasi Cluster Koefisien Elastisitas Harga}
\label{tab:bi_sekda_clustering}
\begin{tabular}{lrrrrrr}
\toprule
\textbf{Cluster} & \textbf{N Prov} & \textbf{$\beta$ (Beras)} & \textbf{$\beta$ (Cabai)} & \textbf{$\beta$ (Bawang)} & \textbf{$\beta$ (Daging)} & \textbf{$\beta$ (Gula)} \\
\midrule
1: Sensitivitas Tinggi & 11 & $-0{,}243$ & $-0{,}156$ & $-0{,}189$ & $-0{,}267$ & $-0{,}142$ \\
2: Sensitivitas Moderat & 13 & $-0{,}152$ & $-0{,}067$ & $-0{,}118$ & $-0{,}168$ & $-0{,}085$ \\
3: Sensitivitas Rendah & 10 & $-0{,}082$ & $-0{,}028$ & $-0{,}052$ & $-0{,}091$ & $-0{,}035$ \\
\bottomrule
\end{tabular}
\end{table}

Cluster 1 (Sensitivitas Tinggi, 11 provinsi) menunjukkan elastisitas negatif yang terbesar di semua variabel harga pangan, mengindikasikan bahwa aksesibilitas kredit di provinsi-provinsi ini sangat \emph{vulnerable} terhadap guncangan harga. Provinsi dalam cluster ini kemungkinan memiliki struktur ekonomi dengan ketergantungan tinggi pada sektor pertanian atau agroindustri, daya beli masyarakat yang rendah dan sensitif terhadap inflasi pangan, dan keterlibatan UMKM yang tinggi dalam \emph{value chain} pangan. Identifikasi cluster ini menjadi prioritas untuk kebijakan stabilisasi harga dan program pengembangan pertanian lokal yang difokus.

Cluster 2 (Sensitivitas Moderat, 13 provinsi) menunjukkan elastisitas harga pada level intermediate, mencerminkan ekonomi yang lebih terdiversifikasi dengan eksposur moderat terhadap dinamika harga pangan.

Cluster 3 (Sensitivitas Rendah, 10 provinsi) menunjukkan elastisitas harga paling kecil dalam magnitude, mencerminkan provinsi dengan basis ekonomi yang lebih urban dan berbasis layanan, daya beli yang relatif lebih tinggi, dan keterlibatan UMKM yang lebih rendah dalam sektor pangan.

Pemetaan spasial dari ketiga cluster disajikan dalam Gambar~\ref{fig:bi_sekda_cluster_maps}, mengidentifikasi konsentrasi geografis dari tipologi-tipologi yang berbeda dan memfasilitasi desain kebijakan regional yang diferensiasi sesuai karakteristik ekonomi lokal.

\begin{figure}[htbp]
\centering
\includegraphics[width=\textwidth]{GAMBAR/BI_SEKDA_05_Cluster_Maps.pdf}
\caption{Peta clustering elastisitas harga koefisien lokal GWR yang menunjukkan 3 cluster sensitivitas: Tinggi (merah), Moderat (hijau), dan Rendah (biru) di 34 provinsi Indonesia.}
\label{fig:bi_sekda_cluster_maps}
\end{figure}

\subsection{Validasi Diagnostik Model GA-GWR dengan Backbone GAT}

Sebagai pelengkap analisis empiris terhadap metodologi GA-GWR, dilakukan evaluasi diagnostik komprehensif pada model GA-GWR dengan \emph{Graph Attention Network} (GAT) sebagai \emph{backbone} untuk pembelajaran bobot spasial adaptif pada data uji BI-SEKDA. Meskipun hasil empiris menunjukkan bahwa GWR dengan kernel Exponential menghasilkan performa prediktif superior, diagnostik residual GA-GWR memberikan wawasan penting tentang karakteristik ketidakcocokan model dan implikasi dari asumsi-asumsi yang mendasarinya.

Hasil diagnostik residual model GA-GWR disajikan dalam Tabel~\ref{tab:bi_sekda_ga_gwr_diagnostics}. Uji Shapiro-Wilk menghasilkan nilai-$p$ = 0{,}0000, menunjukkan penyimpangan signifikan dari normalitas distribusi pada tingkat 5%. Penyimpangan ini dapat dikaitkan dengan ukuran sampel yang besar ($n = 680$) yang membuat uji ini sangat sensitif terhadap deviasi minor dari normalitas yang mungkin tidak material dalam konteks praktis. Uji Breusch-Pagan menghasilkan nilai-$p$ = 0{,}0000, mengindikasikan kehadiran heteroskedastisitas signifikan dalam residual, dengan variansi yang meningkat pada range nilai prediksi yang lebih tinggi, mencerminkan bahwa model menghasilkan prediksi yang kurang akurat pada magnitude kredit yang lebih besar.

Meskipun diagnostik menunjukkan penyimpangan dari asumsi normalitas, distribusi residual tetap menunjukkan pusat pada nilai nol dengan mean observasi sebesar 0{,}0048, mengindikasikan bahwa estimator bersifat \emph{unbiased}. Heteroskedastisitas yang diidentifikasi dapat ditangani melalui beberapa pendekatan: pertama, penggunaan standar error yang adjusted untuk heteroskedastisitas seperti \emph{Huber-White sandwich estimator}; kedua, transformasi logaritmik pada variabel respons untuk memperkecil jarak skala; ketiga, penggunaan \emph{weighted least squares} dengan bobot yang berkorelasi dengan magnitude prediksi.

Inspeksi visual distribusi residual dan plot Q-Q (Gambar~\ref{fig:bi_sekda_ga_gwr_residual}) menunjukkan kehadiran ekor yang lebih berat pada kedua ujung distribusi, konsisten dengan kehadiran nilai-nilai ekstrim dan heteroskedastisitas yang dideteksi melalui uji diagnostik formal. Scatter plot antara nilai prediksi dan residual menunjukkan pola corong (\emph{funnel pattern}) yang melebar seiring peningkatan magnitude prediksi, mengkonfirmasi heterogenitas variansi residual di sepanjang range nilai prediksi.

Secara keseluruhan, analisis empiris pada data BI-SEKDA menunjukkan bahwa metodologi pemodelan spasial berbasis GWR dan GA-GWR, dengan validasi mendalam meliputi uji signifikansi lokal, tipologi spasial melalui clustering, dan diagnostik residual komprehensif, menyediakan instrumen yang valuable untuk mengkaji heterogenitas geografis dalam determinan aksesibilitas kredit UMKM dan mengidentifikasi implikasi kebijakan yang berbasis bukti dan efektif di tingkat regional Indonesia. Keterbatasan jangka waktu observasi (20 bulan), definisi spasial berbasis geografis murni (tanpa konektivitas ekonomi), dan superior performance GWR relatif terhadap GA-GWR membuka peluang penelitian lanjutan dengan data jangka panjang, bobot spasial berbasis ekonomi, dan eksplor asi hibrid antara kernel adaptif dengan pembelajaran neural network.