\chapter{STUDI KASUS}

\section{Catatan Penting}
ISINYA YAA PEMBAHASAN STUDI KASUS KAMUUUU.

\section{Pembahasan}
\subsection{Pembahasan 1}
Dalam membuat tabel, disarankan memakai "longtable" agar tabel nya bisa dipotong halaman. Biar lebih mudah kamu bisa pakai web 
\newline \href{https://www.tablesgenerator.com/latex_tables#google_vignette}{https://www.tablesgenerator.com/latex\_tables\#google\_vignette}. 
\par Contoh input gambar
\begin{figure}[h!]
    \centering
    \includegraphics[width=0.6\linewidth]{GAMBAR/patrick.jpeg}
    \caption{SKRIPSI TU DIKERJAIN}
    \label{fig:skripsi}
\end{figure}
\subsection{Pembahasan 2}
Beberapa catatan penulisan yang wajib diperhatikan sebagai berikut.
\begin{itemize}
    \item Penggunaan kata 'adalah' dan 'merupakan'
    \item Penulisan tanda baca seperti titik yang wajib ada di setiap akhir persamaan
    \item Dalam kalimat '... dihitung menggunakan persamaan berikut.' di akhir kata berikut harus ada 'titik'
    \item Kamu bisa pakai 'begin\{equation\}' atau '\$\$' atau '\$\$ begin\{aligned\} ... end\{aligned\} \$\$' dalam menuliskan persamaan, tinggal pilih mana yang kamu butuhkan. Contoh
     $$\begin{aligned}
            (\bm{A})_{22} &= (\bm{C})_{22} + \min \lbrace (\bm{A})_{2(2-1)},(\bm{A})_{(2-1)2},(\bm{A})_{(2-1)(2-1)} \rbrace \\
             &= (\bm{C})_{22} + \min \lbrace (\bm{A})_{21},(\bm{A})_{12},(\bm{A})_{11} \rbrace \\
             &= 2+\min \lbrace 3,5,2 \rbrace \\
             &= 2+2 =4.
        \end{aligned}$$
    \begin{equation}
                (\bm{A})_{11} = (\bm{C})_{11}.
                \label{eqn:e11}
            \end{equation}
    $$ A = (5,\text{ } 6,\text{ } 5,\text{ } 7,\text{ } 6,\text{ } 6,\text{ } 6,\text{ } 6)
    \hspace{1cm}B = (7,\text{ } 8,\text{ } 6,\text{ } 10,\text{ } 10,\text{ } 10,\text{ } 8,\text{ } 8) $$
    \item Pemanggilan persamaan, tabel, dan gambar, WAJIB menggunakan huruf besar di depan. Contohnya adalah '... dapat dihitung menggunakan Persamaan xx', 'Dari Tabel xx', 'Berdasarkan Gambar xx'
\end{itemize}